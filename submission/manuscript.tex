%% LyX 2.1.4 created this file.  For more info, see http://www.lyx.org/.
%% Do not edit unless you really know what you are doing.
\documentclass[times]{qjrms4}
\usepackage[T1]{fontenc}
\usepackage[latin9]{inputenc}
\setcounter{secnumdepth}{3}
\setcounter{tocdepth}{3}
\usepackage{url}
\usepackage{amsmath}
\usepackage{amssymb}
\usepackage{graphicx}
\usepackage[authoryear]{natbib}

\makeatletter
%%%%%%%%%%%%%%%%%%%%%%%%%%%%%% User specified LaTeX commands.
\newcommand{\nicefrac}[2]{\ensuremath ^{#1}\!\!/\!_{#2}}
\usepackage { fancybox}
\usepackage[export]{adjustbox}

%\usepackage{todonotes}
%\usepackage{afterpage}

%\usepackage[switch]{lineno}
%\linenumbers

\usepackage[colorlinks,bookmarksopen,bookmarksnumbered,citecolor=red,urlcolor=red]{hyperref}

\newcommand\BibTeX{{\rmfamily B\kern-.05em \textsc{i\kern-.025em b}\kern-.08em
T\kern-.1667em\lower.7ex\hbox{E}\kern-.125emX}}

%\usepackage{moreverb}

\usepackage{flushend}

\makeatother

\begin{document}
\title{Numerical Solution of the Conditionally Averaged Equations for Representing Net Mass Flux due to Convection}
\author{Hilary Weller,\affil{a}\corrauth\ William McIntyre\affil{a}}

%\runningheads{H. Weller and W. McIntyre}{Conditional Averaging for Convection}

%\abbrevs{EDMF, Eddy Diffusivity Mass Flux; TVD, Totoal Variation Diminishing; KE, Kinetic Energy; PE, Potential Energy; IE, Internal Energy; RMS; Root Mean Square; } 

\address{
\affilnum{a}Meteorology, University of Reading}
\corraddr{E-mail: <h.weller@reading.ac.uk>}

\runningheads{H. Weller and W. McIntyre}{Conditional Averaging for Convection}

\begin{abstract}
The representation of sub-grid scale convection is a weak aspect of weather and climate prediction models and the assumption that no net mass is transported by convection in parameterisations is increasingly unrealistic entering the grey zone of convection. The solution of conditionally averaged equations of motion is proposed in order to avoid this assumption. Separate momentum, continuity and temperature equations are solved for inside and outside convective plumes which interact via mass transfer terms, drag and by a single common pressure. 

This paper presents stable numerical methods for solving the conditionally averaged equations of motion including large transfer terms between the environmental and plume fluids. Without transfer terms the two fluids are not sufficiently coupled together and solutions diverge. Three transfer terms are presented which couple the fluids together in order to stabilise the model; diffusion of mass between the fluids, drag between the fluids and the most novel; mass transfer when mass converges in one of the fluids. Mass transfer terms are also proposed to mimic a convection parameterisation. The transfer terms are limited and treated implicitly in order to achieve bounded, stable solutions. 

Results are presented of a well resolved rising warm bubble with rising air being transferred to the buoyant fluid. For stability, equations are formulated in advective rather than flux form and solved using bounded finite volume methods. Discretisation choices are made to preserve boundedness and conservation when mass is transferred between fluids.

The formulation of transfer terms in order to represent sub-grid convection is the subject of future work.

%\keywords{Convection, atmpspheric, modelling, parameterisation, grey zone}
\end{abstract}
\maketitle


\section{Introduction}

The representation of sub-grid scale convection is arguably the weakest
aspect of weather and climate prediction models \citep[eg.][]{SLF+10,SAB+13,HPB+14}
and leads to poor predictions of weather and climate in the extratropics
\citep[eg. ][]{LCD+08} and the tropics \citep[Chapter 8,][]{ipcc41}.
The problem gets worse in the grey zone of convection, where convection
is partially resolved and so the assumptions made by most convection
schemes are particularly bad \citep[eg. ][]{GG05}. Two specific assumptions
are identified which we aim to avoid:
\begin{enumerate}
\item Net mass flux by convection; traditional mass flux (and other) convection
schemes assume that convection does not create a net transport of
mass in the vertical \citep[eg. ][]{GR90}. Instead mass is mixed
within each column.
\item Non-equilibrium dynamics; traditional convection schemes ignore effects
due to changes in time of the properties of convection \citep[eg. ][]{KF90}. 
\end{enumerate}
The conditional averaging (or filtering) process for convection was
described by \citet{TWV+18} and involves multiplying each equation
of motion by an indicator function and averaging over a volume (or
applying a different filter). This leads to equations which are similar
to those of a mass flux convection scheme but without the approximation
of zero net mass flux by convection. The conditionally averaged equations
also have transfer terms to transfer mass, momentum, heat and moisture
between the fluids. These terms have a similar role to the closures
for cloud base mass flux and convective entrainment and detrainment. 

There are schemes which account for aspects of non-equilibrium dynamics
\citep[eg. ][]{GG05,YP12,Par14} but fewer that allow net mass flux
due to convection, exceptions being \citet{GG05,KGB07,AW13}. \citet{KGB07,KB08}
extend a mass flux convection scheme to transport mass in the vertical
by creating a source term of the continuity equation due to sub-grid
scale convection. Their approach is not as general or consistent as
that proposed by \citet{TWV+18} and it is also not clear if their
numerical technique will be stable for moderate time-steps. Other
attempts to allow net mass flux by convection \citep[eg.][]{GG05,AW13}
have relied on statistical approximations to define the area fraction
associated with convection rather than on prognostic equations, as
laid out by \citet{TWV+18}. A significant advance is the extended
EDMF scheme \citep{TKP+18} which presents conditionally averaged
equations of motion with different fluids for the environmental and
convective plume, including transport equations for the plume area
fraction. \citet{TKP+18} combine conditional averaging and Reynolds
averaging, presenting transport equations for sub-grid scale variability
in each fluid. However the numerical solutions that they present are
in a single column and they assume that no net mass is transferred
out of the column in order to simplify their numerical solution. In
order to make full use of the extended EDMF scheme, a robust numerical
method is needed to solve conditionally averaged equations for convection
in three dimensions. 

Conditional averaging has been used in other fields for decades; \citet{Dopa77}
described how it could be used for representing intermittent turbulent
flows but it has more commonly been used to represent multiphase flow
\citep[eg. ][]{LB91,GBB+07} with separate fluids for different phases
which share a single pressure. The conditionally averaged Euler equations
with a single pressure and without transfers between the fluids are
in fact ill-posed \citep{SW84} and are usually regularlised by including
coupling between phases such as drag and other relaxation transfers.
Alternatively, \citet{HK84} regularised these equations using multiple
pressures for problems with surface tension. \citet{TEB1x} are also
working on a single column solution of conditionally averaged equations
and show that the incompressible conditionally averaged equations
are unstable without additional coupling between the fluids. 

This paper presents a stable numerical method for solving the conditionally
averaged equations in arbitrary dimensions and proposes transfer terms
that transfer resolved convection into the buoyant fluid and stable
air back into the stable fluid. These transfer terms are not designed
to be used to represent sub-grid scale convection as this would require
more information about sub-grid scale variability. Instead they are
designed to be large source terms that will act to challenge the stability
of the numerical method. We use three techniques to regularise the
multifluid equations; the first is with drag between fluids, the second
is with diffusion between the fluids and the third is more novel and
targets specifically the equal and opposite divergence that can grow
in the individual fluids and hence cannot be controlled by pressure
gradients. 


\section{The Conditionally Averaged Euler Equations}

Traditional mass flux convection schemes solve simplified equations
of motion with temperature, vertical velocity and moisture inside
convective plumes. This is therefore a form of conditional averaging
with variables averaged inside and outside plumes. However we can
take the process further and avoid some of the crude assumptions made
by mass flux schemes such as vanishing convective area fraction and
no net mass flux due to convection. The conditional averaging (or
filtering) process for convection was described by \citet{TWV+18}
and involves multiplying each equation of motion by an indicator function,
$I_{i}$, for a number of different conditions labelled by $i$. $I_{i}$
is one if the condition is true at that location and zero otherwise.
A filter (typically volume average) is then applied and averages for
each condition can be found over each filter region. The volume fraction
in fluid $i$ is defined to be
\begin{equation}
\sigma_{i}=\widetilde{I_{i}}
\end{equation}
where the $\widetilde{\ }$ implies the application of the filter
(or volume average). Density, potential temperature and velocity can
then be defined for each fluid:
\begin{eqnarray}
\sigma_{i}\rho_{i} & = & \widetilde{I_{i}\rho}\\
\sigma_{i}\rho_{i}\theta_{i} & = & \widetilde{I_{i}\rho\theta}\\
\sigma_{i}\rho_{i}\mathbf{u}_{i} & = & \widetilde{I_{i}\rho\mathbf{u}}
\end{eqnarray}
and averages can be found over all fluids, denoted by overbar:
\begin{eqnarray}
1 & = & \sum_{i}\sigma_{i}\\
\overline{\rho} & = & \sum_{i}\sigma_{i}\rho_{i}\\
\overline{\rho\theta} & = & \sum_{i}\sigma_{i}\rho_{i}\theta_{i}\\
\overline{\rho\mathbf{u}_{i}} & = & \sum_{i}\sigma_{i}\rho_{i}\mathbf{u}_{i}.
\end{eqnarray}
Non-linear conditionally averaged terms can be expressed as the products
of conditionally averaged terms but this is not equal to the conditional
average of the non-linear terms. The difference is expressed as a
sub-filter scale flux which can be parameterised:
\begin{eqnarray}
\widetilde{I_{i}\rho\mathbf{u}\theta} & = & \sigma_{i}\rho_{i}\mathbf{u}_{i}\theta_{i}+\mathbf{F}_{\text{SF}}^{\theta_{i}}\\
\widetilde{I_{i}\rho\mathbf{u}\mathbf{u}} & = & \sigma_{i}\rho_{i}\mathbf{u}_{i}\mathbf{u}_{i}+\mathsf{F}_{\text{SF}}^{\mathbf{u}_{i}}.
\end{eqnarray}
The sub-filter scale fluxes are typically due to turbulent motions
within each fluid and will be ignored in this paper. 

The same averaging can be applied to pressure but we will assume that
pressure is uniform across all fluids. The time-scale for equilibration
of pressure across all fluids will be related to the speed of sound
so uniform pressure across fluids may be a good approximation and
also makes numerical solution practical. This is also the approximation
made when using conditional averaging for representing multi-phase
flows \citep[eg.][]{GBB+07}. A single pressure for compressible multiphase
flown is know to lead to an ill-posed problem \citep{SW84} but the
equations can be regularised with some kind of coupling between fluids
which will be discussed in section \ref{sub:transfers}. \citet{TKP+18}
do not assume that the pressure is equal in each fluid but they do
assume that density is equal in both fluids, except where it influences
buoyancy. \citet{TKP+18} also assume that drag is high enough that
horizontal velocities are equal between fluids. 

Applying conditional averaging to the rotating compressible Euler
equations in flux form, assuming uniform pressure between fluids and
ignoring sub-fiter scale fluxes leads to the conditionally averaged
Euler equations:
\begin{eqnarray}
\frac{\partial\sigma_{i}\rho_{i}\mathbf{u}_{i}}{\partial t} & + & \nabla\cdot\left(\sigma_{i}\rho_{i}\mathbf{u}_{i}\mathbf{u}_{i}\right)=-2\sigma_{i}\rho_{i}\boldsymbol{\Omega}\times\mathbf{u}_{i}\label{eq:condMom}\\
 & - & \sigma_{i}\rho_{i}c_{p}\theta_{i}\nabla\pi+\sigma_{i}\rho_{i}\mathbf{g}\nonumber \\
 & + & \sum_{j\ne i}\left(\sigma_{j}\rho_{j}\mathbf{u}_{j}S_{ji}-\sigma_{i}\rho_{i}\mathbf{u}_{i}S_{ij}-\sigma_{i}\sigma_{j}\mathbf{d}_{ij}\right)\nonumber \\
\frac{\partial\sigma_{i}\rho_{i}}{\partial t} & + & \nabla\cdot(\sigma_{i}\rho_{i}\mathbf{u}_{i})=\sum_{j\ne i}\left(\sigma_{j}\rho_{j}S_{ji}-\sigma_{i}\rho_{i}S_{ij}\right)\label{eq:condCont}\\
\frac{\partial\sigma_{i}\rho_{i}\theta_{i}}{\partial t} & + & \nabla\cdot\left(\sigma_{i}\rho_{i}\mathbf{u}_{i}\theta_{i}\right)\label{eq:condTheta}\\
 & = & \sum_{j\ne i}\left(\sigma_{j}\rho_{j}\theta_{j}S_{ji}-\sigma_{i}\rho_{i}\theta_{i}S_{ij}-\sigma_{i}\rho_{i}H_{ij}\right)\nonumber 
\end{eqnarray}
where $\pi=(p/p_{0})^{\kappa}$ is the Exner pressure, $p$ is the
pressure, $p_{0}$ is a reference pressure, $\kappa=R/c_{p}$, $R$
is the gas constant of dry air, $c_{p}$ is the heat capacity of dry
air at constant pressure, $\theta=T/\pi$ is the potential temperature,
$\boldsymbol{\Omega}$ is the rotation rate of the domain, $\mathbf{g}$
is the acceleration due to gravity, $\sigma_{i}\rho_{i}S_{ij}$ is
the transfer rate of mass from fluid $i$ to fluid $j$, $\mathbf{D}_{ij}$
is the drag exerted from fluid $i$ onto fluid $j$ and $H_{ij}$
is the heat transfer from fluids $i$ to $j$. When mass is transferred,
its properties are taken with it which is why $S_{ij}$ appears in
all equations. We then also assume the equation of state for dry air
both globally and for each fluid:
\begin{equation}
p_{0}\pi^{\frac{1-\kappa}{\kappa}}=R\rho_{i}\theta_{i}=R\overline{\rho\theta}=R\sum_{i}\sigma_{i}\rho_{i}\theta_{i}.\label{eq:condState}
\end{equation}
These equations can be expressed in advective form so that the primitive
variables $\mathbf{u}_{i}$ and $\theta_{i}$ are well defined when
$\sigma_{i}$ is zero:
\begin{eqnarray}
\frac{\partial\mathbf{u}_{i}}{\partial t} & + & \mathbf{u}_{i}\cdot\nabla\mathbf{u}_{i}=-2\boldsymbol{\Omega}\times\mathbf{u}_{i}-c_{p}\theta_{i}\nabla\pi+\mathbf{g}\label{eq:condMomAdv}\\
 & + & \sum_{j\ne i}\left(\frac{\sigma_{j}\rho_{j}}{\sigma_{i}\rho_{i}}S_{ji}(\mathbf{u}_{j}-\mathbf{u}_{i})-\mathbf{D}_{ij}\right)\nonumber \\
\frac{\partial\sigma_{i}\rho_{i}}{\partial t} & + & \nabla\cdot(\sigma_{i}\rho_{i}\mathbf{u}_{i})=\sum_{j\ne i}\left(\sigma_{j}\rho_{j}S_{ji}-\sigma_{i}\rho_{i}S_{ij}\right)\label{eq:condCont-1}\\
\frac{\partial\theta_{i}}{\partial t} & + & \mathbf{u}_{i}\cdot\nabla\theta_{i}=\sum_{j\ne i}\left(\frac{\sigma_{j}\rho_{j}}{\sigma_{i}\rho_{i}}S_{ji}(\theta_{j}-\theta_{i})-H_{ij}\right).\label{eq:condThetaAdv}
\end{eqnarray}
Note that if $\sigma_{i}$ is zero, there is a division by zero in
the mass transfer terms in eqns (\ref{eq:condMomAdv}) and (\ref{eq:condThetaAdv})
which leads to an infinite source term when $\theta$ and $\mathbf{u}$
are transferred to an empty fluid. This is appropriate because when
mass is transferred to an empty fluid, the properties should instantaneously
become those of the transferred fluid, the old properties of the empty
fluid should not have any influence. However this infinite source
term will require careful numerical treatment. 


\subsection{Transfers and Exchanges between Fluids\label{sub:transfers}}

The conditionally averaged equations can only become useful for representing
sub-grid scale convection when transfer terms $S_{ij}$, $\mathbf{D}_{ij}$
and $H_{ij}$ are formulated. $S_{ij}$ is particularly important
for moving mass in and out of the fluid related to convection and
will inevitably be associated with sub-grid scale variability of buoyancy
and other properties related to convection. In this paper we do not
propose new closures for how mass moves from the stable to the convectively
active fluid (such as cloud base mass flux). Instead we will use transfer
terms formulated in terms of differential operators that transfer
resolved flow into the convectively active fluid and back again. The
purpose of this is to test the stability, boundedness and conservation
properties of the numerical methods rather than proposing a useful
parameterisation of convection. Throughout we assume $H_{ij}=0$.

We also use the transfer terms to couple the fluids, regularising
the ill-posed equations. 


\subsubsection{Diffusion of $\sigma$\label{sub:diffuseSigma}}

Firstly, a diffusive mass transfer term is created to smooth out steep
changes in $\sigma_{i}$. This term can control numerical oscillations
in $\sigma_{i}$ and mixes the fluids but also plays a similar role
to convective entrainment which mixes convectively active and stable
air. The form used ensures that total mass is not diffused and that
the transfer from $i$ to $j$ is always positive:
\begin{equation}
\sigma_{i}\rho_{i}S_{ij}=\frac{K_{\sigma}}{2}\max\left(\nabla^{2}\left(\sigma_{j}\rho_{j}-\sigma_{i}\rho_{i}\right),\ 0\right)\label{eq:diffusionTransfer}
\end{equation}
where $K_{\sigma}$ is a diffusion coefficient which can be chosen
so to obey stability constraints if the mass transfer term is treated
explicitly in equation (\ref{eq:condCont-1}). 


\subsubsection{Transfers based on Divergence\label{sub:divTransfer}}

In the fully compressible Euler equations, divergence is tightly controlled
by pressure gradients and in low Mach number flows such as atmospheric
dynamics, divergence is small. However in the conditionally averaged
equations without drag or transfers between fluids, there is nothing
to control the divergence in each fluid because there is only one
pressure gradient that acts on all fluids. \citet{TV1x} found the
normal modes of the conditionally averaged equations and noted the
divergent modes in each fluid that are not accompanied by any pressure
anomalies. We will see in section \ref{sec:results} that fluid divergence
can grow leading to large velocities and model instability. One mechanism
to control the  divergence would be to transfer converging fluid to
the other fluid and vice-verca. The transfer terms need to be found
so that, in the absence of mean divergence, no new extrema in $\sigma_{i}\rho_{i}$
are generated. This can be achieved by setting:
\begin{eqnarray}
\sigma_{i}\rho_{i}S_{ij} & = & \frac{1}{2}\max\left(\sigma_{j}\rho_{j}\nabla\cdot\mathbf{u}_{j}-\sigma_{i}\rho_{i}\nabla\cdot\mathbf{u}_{i},\ 0\right)\label{eq:divTransfer}
\end{eqnarray}
assuming two fluids labelled $i$ and $j$. The coefficient of $\frac{1}{2}$
is chosen so that when eqn (\ref{eq:divTransfer}) is re-arranged,
the left hand side is in advective form and the right hand side consists
of divergence averaged over all fluids:
\begin{equation}
\frac{\partial\sigma_{i}\rho_{i}}{\partial t}+\nabla\cdot(\sigma_{i}\rho_{i}\mathbf{u}_{i})=\frac{1}{2}\left(\sigma_{i}\rho_{i}\nabla\cdot\mathbf{u}_{i}-\sigma_{j}\rho_{j}\nabla\cdot\mathbf{u}_{j}\right)\label{eq:contWithDivTransfer}
\end{equation}
\begin{equation}
\implies\frac{\partial\sigma_{i}\rho_{i}}{\partial t}+\mathbf{u}_{i}\cdot\nabla(\sigma_{i}\rho_{i})=-\frac{1}{2}\overline{\rho}\nabla\cdot\overline{\mathbf{u}}\label{eq:contMaterialConserved}
\end{equation}
where $\overline{\rho}\nabla\cdot\overline{\mathbf{u}}=\sum_{i}\sigma_{i}\rho_{i}\nabla\cdot\mathbf{u}_{i}$,
satisfying the criteria that, in the absence of global divergence,
advection of $\sigma_{i}\rho_{i}$ is bounded. This transfer term
is formulated to stabilise the equations rather than to represent
buoyant convection. Solving eqn (\ref{eq:contMaterialConserved})
for $\sigma_{i}\rho_{i}$ rather than eqn (\ref{eq:condCont-1}) means
that, in the absence of mean divergence, $\sigma_{i}\rho_{i}$ is
materially conserved. 

We cannot simply solve eqn (\ref{eq:contMaterialConserved}) when
using transfers based on fluid divergence because we also need to
calculate the mass transfer term (eqn \ref{eq:divTransfer}) explicitly
as it is needed for solving the $\theta_{i}$ and $\mathbf{u}_{i}$
equations. Instead, the term $\sigma_{j}\rho_{j}\nabla\cdot\mathbf{u}_{j}$
of eqn (\ref{eq:divTransfer}) is calculated as:
\begin{equation}
\sigma_{j}\rho_{j}\nabla\cdot\mathbf{u}_{j}=\nabla\cdot(\sigma_{i}\rho_{i}\mathbf{u}_{i})-\mathbf{u}_{i}\cdot\nabla(\sigma_{i}\rho_{i})
\end{equation}
to ensure exact numerical cancelling of the $\nabla\cdot(\sigma_{i}\rho_{i}\mathbf{u}_{i})$
term on either side of the $\sigma_{i}\rho_{i}$ equation (\ref{eq:condCont-1}). 


\subsubsection{Transfers based on buoyancy perturbations\label{sub:buoyancyTransfer}}

This transfer is formulated to mimic buoyant convection rather than
to regularise the equations. Air will rise if buoyancy perturbations
make it lighter than the air above or lighter than the surroundings.
We want to formulate transfer terms as part of PDEs rather than introducing
criteria comparing the buoyancy of grid boxes with surrounding grid
boxes. Therefore we use the Laplacian of $\theta$ to inform us of
positive and negative perturbations. There will be a positive perturbation
if $\nabla^{2}\theta<0$ and vice versa. These transfer terms can
be added to the transfer terms associated with diffusion between the
fluids (eqn (\ref{eq:diffusionTransfer})):
\begin{eqnarray}
S_{01} & = & \begin{cases}
-K_{\theta}\frac{\nabla^{2}\theta_{0}}{\theta_{0}} & \ \text{when}\ \nabla^{2}\theta_{0}<0\\
0 & \ \text{otherwise}
\end{cases}\label{eq:thetaTransfer01}\\
S_{10} & = & \begin{cases}
K_{\theta}\frac{\nabla^{2}\theta_{1}}{\theta_{1}} & \ \text{when}\ \nabla^{2}\theta_{1}>0\\
0 & \ \text{otherwise}
\end{cases}\label{eq:thetaTransfer10}
\end{eqnarray}
where $K_{\theta}$ is a diffusivity. 


\subsubsection{Transfers Based on Horizontal Divergence and Vertical Velocity\label{sub:hdiv_w_Transfer}}

This transfer is also formulated to mimic convection rather than to
regularise the equations. The transfer term moves fluid from fluids
zero to one when fluid zero is converging in the horizontal and rising
and that moves fluid from fluid one to fluid zero when fluid one is
diverging in the horizontal and falling:
\begin{eqnarray}
\sigma_{0}\rho_{0}S_{01} & = & \begin{cases}
-\nabla_{h}\cdot(\sigma_{0}\rho_{0}\mathbf{u}_{0}) & \text{if}\ \nabla_{h}\cdot(\sigma_{0}\rho_{0}\mathbf{u}_{0})<0\\
 & \text{and}\ \mathbf{u}_{0}\cdot\mathbf{g}<0\\
0 & \ \text{otherwise}
\end{cases}\label{eq:divTransfer01}\\
\sigma_{1}\rho_{1}S_{10} & = & \begin{cases}
\nabla_{h}\cdot(\sigma_{1}\rho_{1}\mathbf{u}_{1}) & \text{if}\ \nabla_{h}\cdot(\sigma_{1}\rho_{1}\mathbf{u}_{1})>0\\
 & \text{and}\ \mathbf{u}_{1}\cdot\mathbf{g}>0\\
0 & \ \text{otherwise.}
\end{cases}\label{eq:divTransfer10}
\end{eqnarray}
As with the transfer term associated with buoyancy, this transfer
term can be added to the transfer terms associated with diffusion
and then limited to preserve boundedness.


\subsubsection{Drag in the Momentum Equation\label{sub:drag}}

Drag between fluids is likely to be a large term in any parameterisation
but also has a strong stabilising effect, removing equal and opposite
divergence between the fluids. The drag between fluids is based on
a model for drag on rising bubbles described by \citet{RLD+11}. Assuming
exactly two fluids and remembering that we need $\sigma_{i}\rho_{i}D_{ij}=-\sigma_{j}\rho_{j}D_{ji}$
we can use: 
\begin{equation}
\mathbf{D}_{ij}=\frac{\sigma_{j}}{\rho_{i}}\frac{C_{D}\overline{\rho}}{r_{c}}|\mathbf{u}_{i}-\mathbf{u}_{j}|\left(\mathbf{u}_{i}-\mathbf{u}_{j}\right)\label{eq:dragBubble}
\end{equation}
where $C_{D}$ is a drag coefficient and $r_{c}$ is the radius or
length scale of a region of fluid (which needs to be the same for
both fluids). As $\sigma_{i}$ becomes small in any fluid, we need
$r_{c}$ to become small which increases the drag between fluids.
If we assume a maximum and minimum cloud radius, $r_{\max}$ and $r_{\min}$,
then the cloud radius can take the form
\begin{equation}
r_{c}=\max\left(r_{\min},\ r_{\max}\prod_{i}\sigma_{i}\right).\label{eq:dragRadius}
\end{equation}



\section{Semi-Implicit Numerical Solution Technique}

The equations are discretised and solved using the OpenFOAM library
(\url{https://openfoam.org}) and the full implementation is part
of the AtmosFOAM repository (\url{https://github.com/AtmosFOAM/}).
The spatial discretisation uses standard OpenFOAM operators. 


\subsection{Spatial Discretisation}

The spatial discretisation uses a finite-volume C-grid for an arbitrary
mesh, similar to that described by \citet{WS14} with $\theta_{i}$
, $\sigma_{i}\rho_{i}$ and $\pi$ defined as volumetric mean quantities
(or at cell centres) and normal components of velocity defined on
cell faces. All the meshes used are orthogonal and the focus of this
paper is not spatial discretisation therefore for simplicity the discretisation
is described for orthogonal meshes. The spatial discretisation of
each of the terms of the conditionally averaged Euler equations are
described in turn.


\subsubsection{Reconstruction of velocity fields at cell centres and faces from
face normals}

The prognostic velocity variable is the volume flux across each face:
\begin{equation}
U_{i}=\mathbf{u}_{fi}\cdot\mathbf{S}_{f}
\end{equation}
where $\mathbf{u}_{f}$ is the  velocity at the face and $\mathbf{S}_{f}$
is the face area vector. The face velocity is interpolated from the
cell centre velocity using linear interpolation:
\begin{equation}
\mathbf{u}_{fi}=\lambda\mathbf{u}_{ci}+(1-\lambda)\mathbf{u}_{Ni}\label{eq:linearInterpu}
\end{equation}
where $\mathbf{u}_{ci}$ is the cell centre velocity of the cell that
owns face $f$, $\mathbf{u}_{Ni}$ is the cell centre velocity of
the cell on the other side of face $f$ and $\lambda$ is the linear
interpolation weight. The face area vector, $\mathbf{S}_{f}$ is normal
to the face, has the area of the face and points from the owner cell
to the neighbour cell. The cell centre velocity is reconstructed from
surrounding values of $U_{i}$ using the standard OpenFOAM \texttt{fvc::reconstruct}:
\begin{equation}
\mathbf{u}_{ci}=\left(\sum_{f\in c}\mathbf{\hat{S}}_{f}\mathbf{S}_{f}^{T}\right)^{-1}\sum_{f\in c}U_{i}\mathbf{\hat{S}}_{f}
\end{equation}
where the hat denotes the unit vector and the notation $f\in c$ means
all the faces, $f$ of cell $c$. Note that $\sum_{f\in c}\mathbf{\hat{S}}_{f}\mathbf{S}_{f}^{T}$
is a tensor defined on each cell that depends only on the mesh and
its inverse multiplies the vector $\sum_{f\in c}U_{i}\mathbf{\hat{S}}_{f}$
for each cell. 


\subsubsection{Non-Linear Advection}

The conditionally averaged Euler equations are solved in advective
form so that they are defined where $\sigma_{i}=0$ and so that a
bounded advection scheme can be applied to $\sigma_{i}$. The finite
volume technique most naturally lends itself to solving equations
in flux form rather than advective form and so the non-linear advection
term of the momentum equation is calculated as:
\begin{equation}
\mathbf{u}_{i}\cdot\nabla\mathbf{u}_{i}=\nabla\cdot\left(\mathbf{u}_{i}^{T}\mathbf{u}_{i}\right)-\mathbf{u}_{i}\nabla\cdot\mathbf{u}_{i}.
\end{equation}
This quantity is calculated at cell centres and then linearly interpolated
onto faces. Both terms use Gauss's divergence theorem:
\begin{equation}
\mathbf{u}_{i}\cdot\nabla\mathbf{u}_{i}\approx\frac{1}{V}\left(\sum_{f\in c}\mathbf{u}_{ai}U_{i}-\mathbf{u}_{ci}\sum_{f\in c}U_{i}\right)\label{eq:nonLinearAdvect}
\end{equation}
where $V$ is the cell volume and $\mathbf{u}_{ai}$ is the velocity
interpolated from cell centres to faces using the OpenFOAM linear
upwind advection scheme. The prognostic variable is $U_{i}$ so the
non-linear advection term from eqn (\ref{eq:nonLinearAdvect}) is
linearly interpolated onto faces and then the dot product is taken
with $\mathbf{S}_{f}$. This advection is not bounded and requires
more interpolations and reconstructions than are usually used for
C-grid advection of velocity. It is used because it makes implementation
of C-grid advection straightforward in OpenFOAM.


\subsubsection{Pressure Gradient including the $\theta_{i}$ pre-factor}

The pressure gradient term, $c_{p}\theta_{i}\nabla\pi$, needs to
be calculated on faces in the normal direction to the face. Since
we assume an orthogonal mesh this can simply be calculated as the
difference in pressure between the cells either side of the face:
\begin{equation}
c_{p}\theta_{i}\nabla\pi\cdot\mathbf{S}_{f}=c_{p}\theta_{fi}\nabla_{S}\pi\approx c_{p}\theta_{fi}\frac{\pi_{c}-\pi_{N}}{\delta}\ |\mathbf{S}_{f}|\label{eq:pressureGrad}
\end{equation}
where $\theta_{fi}$ is $\theta_{i}$ linearly interpolated from cell
centres to faces, $\pi_{c}$ and $\pi_{N}$ are the values of Exner
pressure in the cells either side of face $f$ and $\delta$ is the
distance between the cell centres. The interpolation of $\theta$
in this term means that the discretisation uses Lorenz staggering
in the vertical.


\subsubsection{Transfer Terms in the Momentum Equation}

The transfer terms on the right hand side of the momentum equation
(\ref{eq:condMomAdv}) are calculated at cell centres and so need
to be linearly interpolated onto faces and the dot product taken with
$\mathbf{S}_{f}$ in order to calculate the rate of change of $U_{i}$.
This is not done using conservative interpolation so momentum could
be destroyed when mass is transferred. 


\subsubsection{Advection of $\sigma_{i}\rho_{i}$\label{sub:vanLeerContinuity}}

The mass transfer terms that appear in the momentum and $\theta$
equations (\ref{eq:condMomAdv},\ref{eq:condThetaAdv}) involve division
by $\sigma_{i}\rho_{i}$ so for stability, $\sigma_{i}\rho_{i}$ should
remain positive. Therefore $\sigma_{i}\rho_{i}$ should be advected
using a monotonic scheme. However for energy conservation and for
consistency between the continuity and pressure equation (which uses
the continuity equation, see section \ref{sub:Helmholtz}), $\rho_{i}$
should be advected using centred differencing. The advection term
of eqn \ref{eq:condCont-1} is therefore discretised using Gauss's
divergence theorem as:
\begin{equation}
\nabla\cdot(\sigma_{i}\rho_{i}\mathbf{u}_{i})\approx\frac{1}{V}\sum_{f\in c}\sigma_{ia}\rho_{if}U_{f}
\end{equation}
where $\rho_{if}=\lambda\rho_{ic}+(1-\lambda)\rho_{iN}$ using the
same interpolation weights and notation as for eqn (\ref{eq:linearInterpu})
and where $\sigma_{ia}$ is interpolated from cell centre values of
$\sigma_{i}$ to faces using the monotonic OpenFOAM van-Leer advection
scheme:
\begin{equation}
\sigma_{ia}=\sigma_{iu}+\phi\left(\sigma_{if}-\sigma_{iu}\right)
\end{equation}
where $\sigma_{iu}$ is the value of $\sigma_{i}$ in the upwind cell,
$\sigma_{if}$ is the linearly interpolated value and $\phi$ is the
van-Leer limiter function:
\begin{equation}
\phi=\frac{r+|r|}{1+|r|}\ ,\ \ r=2\frac{\left(\mathbf{x}_{d}-\mathbf{x}_{u}\right)\cdot\nabla_{u}\sigma_{i}}{\sigma_{id}-\sigma_{iu}}-1
\end{equation}
where $\mathbf{x}_{u}$ and $\mathbf{x}_{d}$ are the locations of
the upwind and downwind cell centres, $\sigma_{id}$ is the value
of $\sigma_{i}$ in the downwind cell and $\nabla_{u}\sigma_{i}$
is the gradient of $\sigma_{i}$ calculated in the upwind cell using
Gauss's divergence theorem. This discretisation is only monotonic
when $\rho_{i}$ is sufficiently smooth in the direction of flow which
is usually achieved in low Mach number flows. 


\subsubsection{Advection of $\theta_{i}$\label{sub:thetaAdvect}}

For consistency with advection in the momentum equation (eqn \ref{eq:nonLinearAdvect}),
$\theta$ advection is calculated in advective form using finite volume
operators (using Gauss's theorem):
\begin{equation}
\mathbf{u}_{i}\cdot\nabla\theta_{i}\approx\frac{1}{V}\left(\sum_{f\in c}\theta_{ai}U_{i}-\theta_{i}\sum_{f\in c}U_{i}\right)\label{eq:thetaAdvect}
\end{equation}
where $\theta_{a}$ is interpolated from cell centres to faces using
the OpenFOAM van-Leer advection scheme as for $\sigma_{i}$.


\subsection{Time Stepping Algorithm}

The conditionally averaged Euler equations are solved using Crank-Nicholson
time-stepping with no off-centering and with deferred correction of
explicitly solved variables. An inner loop solves the pressure equation
twice with explicitly represented terms of the momentum equation (advection
and Coriolis) updated each iteration. An outer loop additionally solves
the continuity equations for each $\sigma_{i}\rho_{i}$ and the $\theta_{i}$
equations explicitly. All transfer terms are solved using operator
split either explicit or implicit first-order time-stepping. 


\subsubsection{Initialisation and Updates for Consistency\label{sub:consistency}}

Transport equations are solved for $U_{i}$, $\sigma_{i}\rho_{i}$,
$\theta_{i}$ and $\pi$. However this system is over specified because
$\pi$ can be calculated from all of the $\sigma_{i}\rho_{i}$ and
$\theta_{i}$ using the equation of state. To avoid over specification,
only $\mathbf{u}_{i}$, $\sigma_{i}$, $\theta_{i}$ and $\pi$ are
read in at initialisation. To avoid inconsistencies growing, $\pi$
is recalculated from $\sigma_{i}\rho_{i}$ and $\theta_{i}$ at the
end of every time-step using the equation of state (\ref{eq:condState}). 


\subsubsection{Solving the Continuity Equation\label{sub:solveContinuity}}

The first equation to be solved is the continuity equation for each
$\sigma_{i}\rho_{i}$. It is solved using operator splitting, first
advecting $\sigma_{i}\rho_{i}$ and then applying the transfer terms.
Using a TVD advection scheme with a van-Leer limiter (section \ref{sub:vanLeerContinuity}),
$\sigma_{i}\rho_{i}$ is advected using
\begin{equation}
(\sigma_{i}\rho_{i})^{\ell\ell}=(\sigma_{i}\rho_{i})^{n}-\Delta t\ \nabla\cdot\left(\left[(1-\alpha)\rho_{i}^{n}\mathbf{u}_{i}^{n}+\alpha\rho_{u}^{\ell}\mathbf{u}_{i}^{\ell}\right]\sigma_{i}^{n}\right)
\end{equation}
where $n$ represents values at the old time-level, $\Delta t$ is
the time-step and $\alpha$ is the off-centering parameter. For all
the simulations presented, $\alpha=1/2$ is used making the time-stepping
second-order accurate. $\ell$ represents the most up to date value
available for time level $n+1$ and $\ell\ell$ represents the new
value being calculated. Once value $\ell\ell$ is calculated, it is
used for values $\ell$ (values at levels $\ell$, $\ell\ell$ and
$n+1$ share the same computer memory). At the end of the time-step,
values $\ell$ are used as the values for time-level $n+1$.

The updated value of $\sigma_{i}^{\ell}$ is not used in the divergence
because the advection scheme is designed to use only the old value;
it is most accurate and bounded when using $\sigma_{i}^{n}$. 

Next the mass transfer terms are calculated using all of the most
up to date values of all variables. The mass transfer terms described
in equations (\ref{eq:diffusionTransfer}, \ref{eq:thetaTransfer01},
\ref{eq:thetaTransfer10}, \ref{eq:divTransfer01}, \ref{eq:divTransfer10})
are added together and limited in order to keep $\sigma_{i}\rho_{i}$
positive. This is done by limiting the mass transfer:
\begin{equation}
\sigma_{i}\rho_{i}S_{ij}^{\ell\ell}=\min\left((\sigma_{i}\rho_{i})^{\ell}S_{ij}^{\ell},\ ((\sigma_{i}\rho_{i})^{\ell}-\sigma_{\min}\rho_{i}^{\ell})/\Delta t\right)
\end{equation}
 where $\sigma_{\min}=10^{-9}$ is used in the simulations presented
in section \ref{sec:results}. Then the mass transfer is used to update
$\sigma_{i}\rho_{i}$ explicitly with operator splitting
\begin{equation}
(\sigma_{i}\rho_{i})^{\ell\ell}=(\sigma_{i}\rho_{i})^{\ell}+\Delta t\left(\sigma_{j}\rho_{j}S_{ji}-\sigma_{i}\rho_{i}S_{ij}\right)^{\ell}.
\end{equation}



\subsubsection{Solving the $\theta_{i}$ equation\label{sub:solveTheta}}

After the continuity equation, the $\theta_{i}$ equation is solved
using operator splitting; first advecting $\theta_{i}$ using a TVD
advection scheme then applying the mass transfer terms to the advected
$\theta_{i}$. The mass transfer terms are applied implicitly because
they can be very large due to division by $\sigma_{i}\rho_{i}$. The
advection of $\theta_{i}$ is calculated as:
\begin{multline}
\theta_{i}^{\ell\ell}=\theta_{i}^{n}-\Delta t\ \biggl((1-\alpha)\left[\nabla\cdot(\theta_{i}\mathbf{u}_{i})-\theta_{i}\nabla\cdot\mathbf{u}_{i}\right]^{n}\\
+\alpha\left[\nabla\cdot(\theta_{i}\mathbf{u}_{i})-\theta_{i}\nabla\cdot\mathbf{u}_{i}\right]^{\ell}\biggr)\label{eq:thetaAdvectDt}
\end{multline}
where the spatial discretisation is described in section \ref{eq:thetaAdvect}.

The implicit addition of mass transfer terms is formulated to be specific
for having two fluids although it would be straightforward to generalise.
In order to derive the equations for adding the mass transfer terms
to $\theta_{i}$ we will write the $\theta_{i}$ equation as:
\begin{equation}
\theta_{i}^{\ell\ell}=\theta_{i}^{\ell}+\Delta t\sum_{j\ne i}\left(\frac{\sigma_{j}\rho_{j}}{\sigma_{i}\rho_{i}}S_{ji}(\theta_{j}^{\ell\ell}-\theta_{i}^{\ell\ell})\right)\label{eq:thetaAddSource}
\end{equation}
where $\theta_{i}^{\ell}$ in eqn (\ref{eq:thetaAddSource}) is $\theta_{i}^{\ell\ell}$
from eqn (\ref{eq:thetaAdvectDt}). Note values at level $\ell\ell$
are on the left and right hand side making this is an implicit solution.
For $i=0,1$ this can be re-arranged to give:
\begin{eqnarray}
\theta_{0}^{\ell\ell} & = & \frac{\left(1+\Delta t\ T_{01}\right)\theta_{0}^{\ell}+\Delta t\ T_{10}\theta_{1}^{\ell}}{1+\Delta t\ T_{10}+\Delta t\ T_{01}}\\
\theta_{1}^{\ell\ell} & = & \frac{\theta_{1}^{\ell}+\Delta t\ T_{01}\theta_{0}^{\ell\ell}}{1+\Delta t\ T_{01}}
\end{eqnarray}
where we have used the shorthand $T_{ij}=\frac{(\sigma_{i}\rho_{i})^{\ell}}{(\sigma_{j}\rho_{j})^{\ell}}S_{ij}$.
For conservation of internal energy, it is necessary that the values
of $\sigma_{i}\rho_{i}$ from after advection but before the mass
transfer are used in the calculation of $T_{ij}$. This numerical
treatment of the mass transfer terms in the $\theta_{i}$ equations
also ensures boundedness of $\theta_{i}$ ($\theta_{i}^{\ell\ell}$
will remain between $\theta_{i}^{\ell}$ and $\theta_{j}^{\ell}$).


\subsubsection{Diagnosing $\sigma_{i}$\label{sub:diagnoseSigma}}

After the updates of prognostic variables $(\sigma_{i}\rho_{i})$
and $\theta_{i}$, the diagnostic variable $\sigma_{i}$ can be updated.
$\sigma_{i}$ is not used in isolation from $\rho_{i}$ anywhere in
the equations (\ref{eq:condMomAdv}-\ref{eq:condState}). However
$\sigma_{i}$ and $\rho_{i}$ may be needed independently in closure
assumptions, such as the approximation of the drag (below). Firstly,
each $\rho_{i}$ is calculated from the equation of state:
\begin{equation}
\rho_{i}=\frac{p_{0}\pi^{\frac{1-\kappa}{\kappa}}}{R\theta_{i}}=\frac{\overline{\rho\theta}}{\theta_{i}}.
\end{equation}
Then each $\sigma_{i}$ can be calculated:
\begin{equation}
\sigma_{i}=\frac{(\sigma_{i}\rho_{i})}{\rho_{i}}.\label{eq:diagnoseSigma}
\end{equation}
This calculation will guarantee $\sum_{i}\sigma_{i}=1$.


\subsubsection{Momentum and Continuity\label{sub:Helmholtz}}

The momentum and continuity equations are solved simultaneously, creating
a Helmholtz equation for $\pi$. This is done by expressing the volume
flux, $U_{i}$, and the mean density, $\overline{\rho}$, as linear
functions of $\pi$ and substituting these into the continuity equation.
The normal component of the volume flux, $U_{i}$ is expressed as
a linear function of $\pi$ using the momentum equation:
\begin{equation}
U_{i}^{n+1}=U_{i}^{\prime}-\alpha\Delta tc_{p}\theta_{fi}^{\ell}\nabla_{S}\pi^{n+1}\label{eq:volFluxFromMom}
\end{equation}
where the calculation of $c_{p}\theta_{fi}^{\ell}\nabla_{S}\pi^{n+1}$
is defined in equation (\ref{eq:pressureGrad}) and $U_{i}^{\prime}$
is the explicitly calculated part of the volume flux:
\begin{eqnarray}
U_{i}^{\prime} & = & U_{i}^{n}\label{eq:Uprime}\\
 & - & (1-\alpha)\Delta t\left(\left[\mathbf{u}_{i}\cdot\nabla\mathbf{u}_{i}+2\boldsymbol{\Omega}\times\mathbf{u}_{i}\right]\cdot\mathbf{S}_{f}+c_{p}\theta_{fi}\nabla_{S}\pi\right)^{n}\nonumber \\
 & - & \alpha\Delta t\left(\mathbf{u}_{i}\cdot\nabla\mathbf{u}_{i}+2\boldsymbol{\Omega}\times\mathbf{u}_{i}\right)^{\ell}\cdot\mathbf{S}_{f}+\Delta t\mathbf{g}\cdot\mathbf{S}_{f}.\nonumber 
\end{eqnarray}
Equation (\ref{eq:volFluxFromMom}) is multiplied by the linear interpolate
of $\sigma_{i}\rho_{i}$ onto faces and then the sum is taken over
all fluids to get the total mass flux:
\begin{equation}
F^{n+1}=\sum_{i}(\sigma_{i}\rho_{i})_{f}U_{i}^{\prime}-\alpha\Delta tc_{p}\overline{\rho\theta}_{f}\nabla_{S}\pi^{n+1}.\label{eq:fluxFromMom}
\end{equation}
This will be substituted into the divergence term of the continuity
equation once we have described the linear representation of $\rho$
as a function of $\pi$.

In order to derive a Helmholtz equation for $\pi$ using the continuity
equation, the density is expressed as
\begin{equation}
\overline{\rho}=\Psi\pi\label{eq:usePsi}
\end{equation}
where $\Psi$ is the compressibility from the equation of state:
\begin{equation}
\Psi=\overline{\rho}^{\frac{2\kappa-1}{\kappa-1}}\left(\frac{R\overline{\theta}}{p_{0}}\right)^{\frac{\kappa}{\kappa-1}}.\label{eq:Psi}
\end{equation}
Equations (\ref{eq:fluxFromMom}) and (\ref{eq:usePsi}) are substituted
into the continuity equation and Gauss's divergence theorem is used
to calculate the divergence term:
\begin{eqnarray}
\frac{\Psi^{\ell}\pi^{n+1}-\Psi^{n}\pi^{n}}{\Delta t} & + & \frac{\alpha}{V}\sum_{f\in c}\left\{ \overline{\rho}_{f}^{\ell}U_{i}^{\prime}-\alpha\Delta tc_{p}\left(\overline{\rho\theta}_{f}^{\ell}\right)\nabla_{S}\pi^{n+1}\right\} \nonumber \\
 & + & \frac{1-\alpha}{V}\sum_{f\in c}F^{n}=0\label{eq:Helmholtz}
\end{eqnarray}
where $V$ is the cell volume. There are no source terms to this equation
because the source terms transfer mass between fluids and the total
continuity equation is summed over fluids. This is a Helmholtz equation
that can be solved for $\pi^{n+1}$. Back substitutions are then made
to calculate each $U_{i}^{n+1}$ using equation (\ref{eq:volFluxFromMom}). 


\paragraph{Applying Drag and Mass Transfer to the Momentum Equation}

The transfer terms of equation (\ref{eq:condMomAdv}) can be applied
after the solution of the Helmholtz equation because they do not directly
influence the pressure. They are applied implicitly, first-order with
operator splitting with a simultaneous solution for two fluids, $i$
and $j$. We assume that $U_{i}^{\ell}$ is the volume flux predicted
by the back substitution after the Helmholtz equation and $U_{i}^{\ell\ell}$
is the solution of $U_{i}$ after implicit application of the source
terms:
\begin{eqnarray}
U_{i}^{\ell\ell} & = & U_{i}^{\ell}-\Delta t\ \left(\frac{\sigma_{j}\rho_{j}}{\sigma_{i}\rho_{i}}S_{ji}+\sigma_{j}\frac{C_{D}\overline{\rho}}{r_{c}\rho_{i}}|\mathbf{u}_{i}-\mathbf{u}_{j}|\right)U_{i}^{\ell\ell}\nonumber \\
 & + & \Delta t\left(\frac{\sigma_{j}\rho_{j}}{\sigma_{i}\rho_{i}}S_{ji}+\sigma_{j}\frac{C_{D}\overline{\rho}}{r_{c}\rho_{i}}|\mathbf{u}_{i}-\mathbf{u}_{j}|\right)U_{j}^{\ell\ell}
\end{eqnarray}
Using the same block implicit solution technique as was described
in section \ref{sub:solveTheta}, $U_{i}^{\ell\ell}$ can be calculated
using:
\begin{eqnarray}
U_{0}^{\ell\ell} & = & \frac{\left(1+\Delta t\ T_{01}\right)U_{0}^{\ell}+\Delta t\ T_{10}U_{1}^{\ell}}{1+\Delta t\ T_{10}+\Delta t\ T_{01}}\label{eq:Utransfer0}\\
U_{1}^{\ell\ell} & = & \frac{U_{1}^{\ell}+\Delta t\ T_{01}U_{0}^{\ell\ell}}{1+\Delta T_{01}}\label{eq:Utransfer1}
\end{eqnarray}
where for $U_{i}$ we use the shorthand 
\begin{equation}
T_{ij}=\frac{\sigma_{i}\rho_{i}}{\sigma_{j}\rho_{j}}S_{ij}+\frac{\sigma_{i}}{\rho_{j}}\frac{C_{D}\overline{\rho}}{r_{c}}|\mathbf{u}_{j}^{\ell}-\mathbf{u}_{i}^{\ell}|.
\end{equation}
as with the numerical method for applying the mass transfer terms
to the $\theta_{i}$ equations, this technique ensures that the $U_{i}$
remain bounded and the use of the values of $\sigma_{i}\rho_{i}$
from before mass transfer in the calculation of $T_{ij}$ gives momentum
conservation on transfer.


\subsubsection{Overview of the Solution Algorithm}
\begin{enumerate}
\item An outer loop is executed twice consisting of:

\begin{enumerate}
\item Solve for $\sigma_{i}\rho_{i}$ as described in section \ref{sub:solveContinuity}.
\item Solve for $\theta_{i}$ as described in section \ref{sub:solveTheta}.
\item Update $\sigma_{i}$ as described in section \ref{sub:diagnoseSigma}.
\item An inner loop is executed twice consisting of the calculations described
in section \ref{sub:Helmholtz}:

\begin{enumerate}
\item Update each $U_{i}^{\prime}$ using eqn (\ref{eq:Uprime}) which consists
of all of the terms of the momentum equation excluding the pressure
gradient term and excluding transfer terms.
\item Calculate the compressibility, $\Psi$, from eqn (\ref{eq:Psi}).
\item Construct and solve the Helmholtz eqn (\ref{eq:Helmholtz}) for $\pi$.
\item Back substitute, adding the pressure gradient term to $U_{i}^{\prime}$
to get $U_{i}^{\ell}$ using eqn (\ref{eq:volFluxFromMom}).
\item Add the transfer terms to $U_{i}^{\ell}$ to get $U_{i}^{n+1}$ using
eqn (\ref{eq:Utransfer0},\ref{eq:Utransfer1}).
\end{enumerate}
\end{enumerate}
\item Update $\Psi$ and $\pi$ at the end of each time-step (section \ref{sub:consistency}).
\end{enumerate}

\section{Results\label{sec:results}}

No test cases exist for numerical solutions of the conditionally averaged
Euler equations and so variations of the rising bubble test case \citep{BF02}
for the non-hydrostatic, compressible Euler equations are used. If
the conditions in each fluid are initially identical, then the solution
should evolve exactly like the single fluid equations with an additional
advected tracer for the fluid fraction. This is therefore used as
a first test of the numerical method. Tests are next formulated with
different initial conditions in each fluid in order to check that
the solution maintains stability, boundedness and some conservation
properties. Finally, tests are created with fluid one initially empty
and mass is transferred in. The solution should evolve exactly as
the single fluid case because the initial conditions in fluid one,
with no mass, should be irrelevant. 

The dry, warm rising bubble test case of \citet{BF02} consists of
a two dimensional vertical slice of height 10\,km and width 20\,km
initially at rest with a surface pressure of 1000\,mb, an initially
uniform potential temperature of 300\ K. The initial pressure is
in discrete hydrostatic balance with this uniform potential temperature.
A warm perturbation: 
\begin{equation}
\theta^{\prime}=2\cos^{2}\frac{\pi L}{2}\label{eq:thetaPerturb}
\end{equation}
 is added for $L<1$ where $L=\sqrt{\left(\frac{x-x_{c}}{x_{r}}\right)^{2}+\left(\frac{z-z_{c}}{z_{r}}\right)^{2}}$,
$x_{c}=10\ \text{km}$, $z_{c}=2\ \text{km}$ and $x_{r}=z_{r}=2\ \text{km}$.
100\,m grid spacing is used and for all simulations presented a time-step
of 2\,s is used. Regardless of the initial conditions and transfers
between fluids, $\sigma$ should remain bounded between zero and one
and the potential temperature should remain bounded by its initial
bounds. 


\subsection{Two Identical Fluids}

First the warm rising bubble of \citet{BF02} is simulated with the
fluid divided into two fluids with identical initial conditions in
each fluid. No transfers or exchanges between fluids are used. Two
different initial fluid fractions are used as shown at the top of
figure \ref{fig:identicalParts}. These should not influence the evolution
of other variables. The two initial $\sigma$ distributions are
\begin{eqnarray}
\text{symmetric:}\ \sigma & = & \begin{cases}
1 & \ \text{if}\ \biggl|\mathbf{x}-\left(\begin{array}{c}
0\\
2
\end{array}\right)\text{km}\biggr|<2\ \text{km}\\
0 & \ \text{otherwise},
\end{cases}\\
\text{asymmetric:}\ \sigma & = & \begin{cases}
1 & \ \text{if}\ \biggl|\mathbf{x}-\left(\begin{array}{c}
2\\
5
\end{array}\right)\text{km}\biggr|<2\ \text{km}\\
0 & \ \text{otherwise}.
\end{cases}
\end{eqnarray}
The distributions of $\sigma$, $\theta_{i}$ and the velocity in
each fluid after 1000\,s are shown at the bottom of figure \ref{fig:identicalParts}.
$\theta$ and the velocity have remained identical in each fluid and
$\sigma$ has been advected by the flow without any undershoots or
overshoots. The presence of the $\sigma$ field does not influence
the evolution of the velocity or potential temperature in each fluid,
as expected. 

\begin{figure*}
\includegraphics[width=1\linewidth]{identicalParts}

\caption{Initial fluid fraction and $\theta_{i}$ (top) for simulations with
identical properties in each fluid and properties after 1000\,s (bottom).
The vectors and contours for $\theta_{i}$ and $\mathbf{u}_{i}$ for
each fluid are identical. \label{fig:identicalParts}}
\end{figure*}


The stability of the model is demonstrated by plotting the total energy
and the different types of energy in figure \ref{fig:energy1part}.
The left hand side shows normalised kinetic, potential, internal and
total energy changes for the model with a single fluid using van-Leer
advection. The various energies are defined in cells as:
\begin{eqnarray}
\text{KE} & = & \frac{1}{2}\sum_{i}\sigma_{i}\rho_{i}|\mathbf{u}_{ci}|^{2}\\
\text{PE} & = & -\mathbf{g}\cdot\mathbf{x}\sum_{i}\sigma_{i}\rho_{i}\\
\text{IE} & = & c_{v}\pi\sum_{i}\sigma_{i}\rho_{i}\theta_{i}\\
E & = & \text{KE}+\text{PE}+\text{IE}
\end{eqnarray}
and totals are calculated by integrating over space. The normalisation
and calculation of changes is calculated for energy XE as:
\begin{equation}
\widetilde{\text{\text{XE}}}=\frac{\text{\text{XE}}-\text{\text{XE}}(t=0)}{E(t=0)}.
\end{equation}
The dashed lines in figure \ref{fig:energy1part} show negative values
and the solid lines show positive values. In the first part of the
simulation, the single fluid simulation shows internal and potential
energy oscillating in phase with each other, showing nearly conservative
transfers between internal and potential energy. Throughout the simulation
the kinetic energy increases as the rising bubble accelerates while
the total energy decreases monotonically due to stable, dissipative
nature of the model. Part of this dissipation is due to the dissipative
advection of velocity and potential temperature. A simulation is also
run using centred, linear differencing for advection and the total
energy is shown in the right hand of figure \ref{fig:energy1part}.
This simulation looses energy less slowly and the energy loss is no
longer monotonic. The results of this simulation are noisy but stable
with overshoots and undershoots of temperature (not shown).

The accuracy of the energy conservation in figure \ref{fig:energy1part}
appears to be good partly because the energy changes are divided by
a large number -- the total initial energy, including unavailable
energy. The initial potential energy, which is mostly unavailable,
is 31444\ Joules and the initial internal energy is 141446\ Joules,
making the total initial energy 172891\ Joules. A fairer accuracy
estimate might be to normalise with the available potential energy.
This has not been calculated. Instead we could compare with the energy
of the initial warm bubble. The warm bubble contains 24.5\ Joules
of additional internal energy in comparison to the stably stratified
state. If we were to normalise the energy changes with this value
then they would be 7057 times bigger, making a normalised change of
$10^{-6}$ close to a change of 0.01.

The total energy for the the simulations with two identical fluids
with symmetric and asymmetric distributions of $\sigma$ are shown
in figure \ref{fig:energy1part}. This confirms that the presence
of more than one identical fluid does not influence the energy conservation.
In fact the solutions with two identical fluids are identical to the
solution with one fluid to within machine precision. 

\begin{figure*}
\includegraphics[width=1\linewidth]{energy1part}

\caption{Normalised changes in kinetic, potential, internal and total energy
for the rising bubble test case for the model with a single fluid
and the normalised total energy change for models with one fluid with
different advection schemes and for a model with two identical fluids
and no mass transfer. Solid lines show positive changes and dashed
lines show negative changes.\label{fig:energy1part}}
\end{figure*}



\subsection{Different Initial Conditions in each Fluid}

Once each fluid can have different properties, the behaviour of the
solutions changes stable solutions may not exist. The total solution
is close to divergence free because compressibility is small in this
low Mach number regime. However with only one pressure to control
the divergence in two fluids, the divergence in each fluid can be
large. We will therefore initialise the equation to force different
velocities and hence divergence in each fluid. We do not have an analytic
solution for this case but we seek stable solutions and we test energy
conservation since energy is conserved in the continuous equations
in the absence of transfers between the fluids.

In order to simulate two fluids with different properties occupying
the same location, we set $\sigma_{1}=\frac{1}{2}$ in a circle with
warm air only in fluid 1 and initially stationary flow in each fluid:
\begin{eqnarray}
\sigma_{1} & = & \begin{cases}
\frac{1}{2} & \ \text{if}\ \biggl|\mathbf{x}-\left(\begin{array}{c}
0\\
2
\end{array}\right)\text{km}\biggr|<2\ \text{km}\\
0 & \ \text{otherwise},
\end{cases}\\
\sigma_{0} & = & 1-\sigma_{1}\\
\theta_{0} & = & 300\ \text{K}\\
\theta_{1} & = & 300\ \text{K}\ +\theta^{\prime}.
\end{eqnarray}
We assume no mass flux and no drag between fluids. The initial conditions
for $\sigma_{1}$ and $\theta_{1}$ are shown in figure \ref{fig:diffuse1_noDrag}
along with the solutions after 100, 200 and 290\ s. $\theta_{0}$
remains identically 300\,K throughout the simulation, as expected.

\begin{figure*}
\includegraphics[width=1\linewidth]{diffuse1_noDrag}

\caption{Zoomed in initial conditions and results of a simulation with a buoyant
perturbation in fluid 1 and no mass transfer. $\sigma_{1}$ is shaded,
$\theta_{1}$ is contoured every 0.2\,K, $\mathbf{u}_{0}$ is shown
by black vectors and $\mathbf{u}_{1}$ by red vectors. \label{fig:diffuse1_noDrag}}
\end{figure*}


The buoyancy perturbation in fluid one makes fluid one rise which
raises the pressure above the bubble which forces fluid zero downwards.
Consequently total divergence is controlled but the divergence in
each fluid grows and hence the velocities become large. The advection
scheme is only bounded for Courant numbers less than 0.5. At 276\,s,
the mean Courant number becomes larger than 0.5 and oscillations grow
in $\sigma_{1}$. The solution diverges at $t=296\ \text{s}$. A stable
simulation could be maintained for longer by using a smaller time-step
or by treating advection implicitly but we should first consider the
physical meaning of this test case. The situation with $\sigma_{1}\sim\frac{1}{2}$
implies that both fluids are present at scales down to the grid-scale
which implies that there will be a large surface area between the
two fluids and so they are likely to exchange mass and momentum. We
therefore consider coupling the two fluids using mass exchanges or
drag between the fluids. 

We first add drag between the fluids but no mass transfer. The drag
takes the form described in section \ref{sub:drag} with length scales
$r_{\min}=1\ \text{m}$ (a minimum is needed to avoid division by
zero) and $r_{\max}=2000\ \text{m}$. A large value of $r_{\max}$
has been chosen so that the drag is low where neither $\sigma$ is
vanishing which allows some variation of velocity between fluids.
The results at $t=1000\ \text{s}$ are shown in figure \ref{fig:diffuse1_dragDiffuse}.
$\theta_{0}$ remains identically 300\,K throughout. The drag has
stabilised the solution and the two fluid velocities (in black and
red) are not identical. 

\begin{figure*}
\includegraphics[width=1\linewidth]{diffuse1_dragDiffuse}

\caption{Results at $t=1000\ \text{s}$ of simulations with a buoyant perturbation
in fluid 1. $\sigma_{1}$ is shaded, $\theta_{1}$ and $\theta_{2}$
are contoured every 0.2\,K, $\mathbf{u}_{0}$ is shown by black vectors
and $\mathbf{u}_{1}$ by red vectors. Initially $\sigma=\frac{1}{2}$
in circle near the ground (as in fig \ref{fig:diffuse1_noDrag}).
 \label{fig:diffuse1_dragDiffuse}}
\end{figure*}


Next the fluids are coupled by adding diffusion as described in section
\ref{sub:diffuseSigma}. This is similar to convective entrainment
and detrainment due to turbulence. The results using a diffusion coefficient
of $K_{\sigma}=200\ \text{m}^{2}\text{s}^{-1}$ are shown in figure
\ref{fig:diffuse1_dragDiffuse}. The diffusion is treated explicitly
and this diffusion coefficient is well below the stability limit.
Once mass is transferred then temperature and momentum are transferred
too so the temperature in fluid 1 no longer remains 300\,K. The diffusion
stabilises the solution and the results are different from the result
with drag between the fluids. 

Divergence transfer -- including mass transfer between the fluids
in order to avoid fluid divergence (section \ref{sub:divTransfer})
also stabilises the solution (bottom left of figure \ref{fig:diffuse1_dragDiffuse}).
Local divergence is removed and so $\sigma_{i}$ remains bounded and
the solution is stable. Temperature and momentum are transferred to
the other fluid but they remain different in each fluid. This stabilisation
technique is free from parameters. 

The total energy changes for all of the simulations with warm air
in a diffuse fluid 1 are shown in the bottom right of figure \ref{fig:diffuse1_dragDiffuse}.
The energy diverges for the unstable case with no transfers. The other
simulations are stabilised either by mass transfers or by momentum
transfer (drag). Both of these destroy kinetic energy and so we expect
to see the energy decrease monotonically for the stabilised simulations.
The simulation with the low diffusion coefficient ($K_{\sigma}=200\ \text{m}^{2}\text{s}^{-1}$)
stabilises the model after around 350\,s but this looks unreliable
as energy increases before this point. A larger diffusion coefficient,
($K_{\sigma}=800\ \text{m}^{2}\text{s}^{-1}$) stabilises the model
more effectively. (The stability limit is $K_{\sigma}\Delta t/\Delta x^{2}<\frac{1}{2}$
so for $\Delta t=2\text{s}$ and $\Delta x=100\text{m}$ the stability
limit is $K_{\sigma}=2500\ \text{m}^{2}\text{s}^{-1}$.) The simulation
is also stabilised by using drag between the fluids. The low drag
coefficient of $C_{D}=1$, $r_{\min}=1\ \text{m}$ and $r_{\max}=2000\ \text{m}$
leads more rapid energy loss than any of the other effective stabilisation
methods. A high drag coefficient ($C_{D}=10^{6}$) is also used to
check the stability of the implicit treatment of drag. The simulation
stabilised by mass transfer based on divergence looses energy monotonically,
more slowly than the other stabilisations. 


\subsection{Transfers Between Fluids to mimic convection parameterisation}

For the conditionally averaged equations to be useful for convection,
there must be transfer terms between the fluids based on properties
related to convection. These have a similar role to the closure approximations
used by conventional convection schemes and could take the same form.
Alternatively they could depend on approximations of sub-grid scale
variability. Here we test mass transfers between fluids based on buoyancy
anomalies and based on horizontal divergence. We are not assuming
any sub-grid scale variability so these are not representative of
convection but we need to demonstrate that the numerical method is
stable in the presence of large transfers between fluids and we choose
transfer terms that lead to warm rising air in fluid one and the stable
and descending air in fluid zero. We examine stability, energy conservation
and comparison with the single fluid case. These simulations start
with no fluid one (all the mass initially in fluid zero). Whenever
mass is transferred into fluid one it takes its properties with it
so the solution should be identical to the single fluid case and energy
should be conserved.


\subsubsection{Transfers based on buoyancy perturbations}

We test the numerical solution using the mass transfer terms associated
with buoyancy perturbations from eqns (\ref{eq:thetaTransfer01},\ref{eq:thetaTransfer10}).
To test that warm air is transferred from fluid zero to fluid one,
we initialise the simulation with $\sigma_{0}=0$ everywhere and the
warm perturbation in fluid zero only. The solutions for $\sigma_{1}$,
$\theta_{0,1}$ and $\mathbf{u}_{0,1}$ after 1000\,s are shown in
figure \ref{fig:massTransfer} using diffusivity $K_{\theta}=10^{6}\ \text{m}^{2}\text{s}^{-1}$.
Diffusion between fluids of $K_{\sigma}=200\ \text{m}^{2}\text{s}^{-1}$
is also used to maintain a smooth solution. Figure \ref{fig:massTransfer}
confirms that $\theta_{i}$ and $\mathbf{u}_{i}$ are similar in each
fluid and also similar to the solutions in figure \ref{fig:identicalParts}.
It also shows that fluid has been transferred to fluid one where there
are warm anomalies. The bottom left of figure \ref{fig:massTransfer}
shows the solution using mass transfer associated with buoyancy perturbations
combined with divergence transfer to stabilise rather than diffusion
between fluids. The velocity and $\theta$ results are identical but
the $\sigma$ field has sharper gradients as there is no diffusion
between the fluids. The $\ell_{2}$ errors shown in figure \ref{fig:massTransfer}
show the root mean square difference between the single fluid $\theta$
and the mean $\theta$ across all partitions ($\sum_{i}\sigma_{i}\rho_{i}\theta_{i}/\sum_{i}\sigma_{i}\rho_{i}$)
normalised by the root mean square single fluid $\theta$. Fluid one
is initialised with no mass and without a warm bubble. The zero mass
in fluid one means that once mass is transferred to fluid one it should
have identical properties to fluid zero. This does not happen exactly;
the RMS errors are low. These RMS errors can be compared with the
RMS differences between the simulations initialised with identical
properties in both fluids and the single fluid in figure \ref{fig:identicalParts}.
These have RMS differences of $7.7\times10^{-14}$. Therefore the
simulation of a zero mass fluid with different initial conditions
is introducing numerical error. Less numerical error is introduced
when using diffusion stabilisation than divergence stabilisation. 

\begin{figure*}
\includegraphics[width=1\linewidth]{massTransfer}

\caption{Rising bubble solutions of $\sigma_{1}$, $\theta_{0,1}$ and $\mathbf{u}_{0,1}$
after 1000\,s with mass transfers based on $K_{\theta}\nabla^{2}\theta/\theta$
and based on $\nabla_{h}\cdot\mathbf{u}$ and $\mathbf{u}\cdot\mathbf{g}$.
The top simulations use $K_{\sigma}=200\ \text{m}^{2}\text{s}^{-1}$
and the bottom simulations use $K_{\sigma}=0$ and divergence transfer
to stabilise. The $\ell_{2}$ error norms are the normalised root
mean square difference between the single fluid $\theta$ and the
mean $\theta$ across all partitions ($\sum_{i}\sigma_{i}\rho_{i}\theta_{i}/\sum_{i}\sigma_{i}\rho_{i}$).
\label{fig:massTransfer}}
\end{figure*}


The changes in energy for both solutions with mass transfer based
on buoyancy perturbations are shown in figure \ref{fig:massTransferEnergy}.
The energy loss is very similar to the single fluid case (figure \ref{fig:energy1part})
with a slight discrepancy for the divergence transfer stabilisation
simulation. 

\begin{figure}
\noindent \begin{centering}
\includegraphics[width=1\linewidth]{massTransferEnergy}
\par\end{centering}

\caption{Normalised energy changes for the rising bubble solutions with mass
transfer based on $K_{\theta}\nabla^{2}\theta/\theta$ and based on
$\nabla_{h}\cdot\mathbf{u}$ and $\mathbf{u}\cdot\mathbf{g}$. \label{fig:massTransferEnergy}}
\end{figure}



\subsubsection{Transfers Based on Horizontal Divergence and Vertical Velocity}

The results of the simulation using mass transfer based on horizontal
divergence and vertical velocity from eqns (\ref{eq:divTransfer01},\ref{eq:divTransfer10})
are shown on the right side of figure \ref{fig:massTransfer}. The
top right shows the simulation stabilised with diffusion between fluids
with $K_{\sigma}=200\ \text{m}^{2}\text{s}^{-1}$ and the bottom right
shows the solution stabilised with divergence transfer. Again, since
fluid one is initially empty, its initial properties should not influence
the final solution. Instead, properties are transferred from fluid
zero and the two fluids evolve in the same way and the solutions are
very similar to the single fluid simulation, with normalised RMS differences
of $9.5\times10^{-7}$ and $1.1\times10^{-5}$. Using mass transfer
based on horizontal divergence leads to larger differences from the
single fluid case but more mass is transferred so this is not a disadvantage
of mass transfer based on horizontal divergence. However transfer
based on horizontal divergence leads to transfer of the fluid that
is behind the warm air and not the warm air itself. The warm air itself
actually initially expands to satisfy the perfect gas law. 

Using mass transfer based on horizontal divergence, the drop in energy
is again very similar to the single fluid case (comparing figures
\ref{fig:energy1part} and \ref{fig:massTransferEnergy}) with larger
discrepancies for the simulation stabilised by divergence transfer. 


\section{Summary, Conclusions and Further Work}

A stable numerical method is presented for solving the conditionally
averaged equations of motion for representing atmospheric convection
with two fluids, one to represent stable, environmental air and the
other to represent buoyant plumes. This builds on traditional mass
flux schemes by solving equations for momentum, mass and temperature
both inside and outside the plumes and so net mass transfer by convection
can be represented. Our numerical method would also be suitable for
the similar extended EDMF scheme \citep{TKP+18} and would allow net
mass transport by convection. Transfers of mass between the fluids
are proposed for two purposes:
\begin{enumerate}
\item The conditionally averaged equations with a single pressure are ill-posed
so transfer terms are formulated to ensure that the different fluids
remain sufficiently close.
\item Mass transfer terms are proposed so that well resolved convection
is transferred to the buoyant fluid and stable or sinking air is transferred
to the stable fluid. These are based on buoyancy anomalies (measured
by the Laplacian of $\theta$) and based on horizontal divergence.
\end{enumerate}
Three types of transfer terms are proposed to couple the two fluids
together; one which diffuses mass between the fluids, one which creates
drag between them and one which moves converging air into the other
fluid. The first two have parameters which must be set in order to
ensure stability. The transfer term that moves converging air into
the other fluid is parameter free and ensures that the fraction in
each fluid, $\sigma_{i}$ is bounded and materially conserved. 

The transfer terms are applied explicitly to the individual partition
continuity equations (the transport equations for $\sigma_{i}\rho_{i}$)
and limited to avoid negative mass in each fluid. These mass transfer
terms also appear in the momentum and temperature equations since
the transferred mass takes its other properties with it. In these
equations the transfer terms are treated implicitly as they can be
very large. The numerical treatment of these transfer terms ensures
boundedness and mass, momentum and internal energy conservation on
transfer.

A semi-implicit finite volume method is used to solve the equations
of motion in advective form which ensures boundedness of the fluid
fraction, $\sigma_{i}$, and enables large Courant numbers with respect
to the speed of sound. This aspect of the solution entails substituting
both (or all) momentum equations into the continuity equation rather
than just one as is done in semi-implicit weather forecast models.
Without this numerical treatment, net mass transport by convection
would trigger acoustic waves that are not handled by the implicit
part of the model which would lead to instability for moderate time-steps. 

Results are presented of a well resolved rising warm bubble with warm
air being transferred to the buoyant fluid. Without transfer terms
the model is unstable since the divergence in each fluid is not controlled.
The model is stable with good energy conservation properties using
both mass transfer as stabilisation and mass transfer to represent
convection. 

The formulation of transfer terms in order to represent sub-grid convection
is the subject of future work. These transfer terms should depend
on sub-grid variability of the primitive variables in each fluid.


\section*{Acknowledgements}

Many thanks to Peter Clark, John Thuburn and Chris Holloway for valuable
discussions and proof reading. Thanks to the NERC/Met Office Paracon
project. We acknowledge funding from the RevCon Paracon project NE/N013743/1.

\bibliographystyle{abbrvnat}
\bibliography{numerics}

\end{document}
