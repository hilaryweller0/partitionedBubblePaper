%% LyX 2.2.3 created this file.  For more info, see http://www.lyx.org/.
%% Do not edit unless you really know what you are doing.
\documentclass[times]{qjrms4}
\usepackage[T1]{fontenc}
\usepackage[latin9]{inputenc}
\setcounter{secnumdepth}{3}
\setcounter{tocdepth}{3}
\usepackage{color}
\usepackage{url}
\usepackage{amsmath}
\usepackage{amssymb}
\usepackage{graphicx}
\usepackage[authoryear]{natbib}

\makeatletter
%%%%%%%%%%%%%%%%%%%%%%%%%%%%%% User specified LaTeX commands.
\newcommand{\nicefrac}[2]{\ensuremath ^{#1}\!\!/\!_{#2}}
\usepackage { fancybox}
\usepackage[export]{adjustbox}

%\usepackage{todonotes}
%\usepackage{afterpage}

\usepackage[switch]{lineno}

\usepackage[colorlinks,bookmarksopen,bookmarksnumbered,citecolor=red,urlcolor=red]{hyperref}

\newcommand\BibTeX{{\rmfamily B\kern-.05em \textsc{i\kern-.025em b}\kern-.08em
T\kern-.1667em\lower.7ex\hbox{E}\kern-.125emX}}

%\usepackage{moreverb}

\usepackage{flushend}

\makeatother

\begin{document}
\linenumbers 
\title{Numerical Solution of the Conditionally Averaged Equations for Representing Net Mass Flux due to Convection}
\author{Hilary Weller,\affil{a}\corrauth\ William McIntyre\affil{a}}

%\abbrevs{EDMF, Eddy Diffusivity Mass Flux; TVD, Totoal Variation Diminishing; KE, Kinetic Energy; PE, Potential Energy; IE, Internal Energy; RMS; Root Mean Square; } 

\address{
\affilnum{a}Meteorology, University of Reading}
\corraddr{E-mail: <h.weller@reading.ac.uk>}

\runningheads{H. Weller and W. McIntyre}{Conditional Averaging for Convection}

\begin{abstract}
The representation of sub-grid scale convection is a weak aspect of weather and climate prediction models and the assumption that no net mass is transported by convection in parameterisations is increasingly unrealistic {\color{blue} as models enter} the grey zone, {\color{blue} partially resolving} convection. The solution of conditionally averaged equations of motion is proposed in order to avoid this assumption. Separate continuity, temperature and momentum equations are solved for inside and outside convective plumes which interact via mass transfer terms, drag and by a single common pressure. {\color{blue} This is not a convection scheme that can be used with an existing dynamical core -- this requires a whole new model.}

This paper presents stable numerical methods for solving the conditionally averaged equations of motion including large transfer terms between the environmental and plume fluids. Without transfer terms the two fluids are not sufficiently coupled together and solutions diverge. {\color{blue} Two} transfer terms are presented which couple the fluids together in order to stabilise the model: diffusion of mass between the fluids (similar to turbulent entrainment) and drag between the fluids. Transfer terms are also proposed to {\color{blue} move buoyant convecting air into the plume fluid and vice-versa as would be needed to represent sub-grid-scale convection. These transfers mimic  cloud base mass flux and cloud top detrainment}. The transfer terms are limited {\color{blue}(clipped in size)} and treated implicitly in order to achieve bounded, stable solutions. 

Results are presented of a well resolved rising warm bubble with rising air being transferred to the buoyant fluid. For stability, equations are formulated in advective rather than flux form and solved using bounded finite volume methods. Discretisation choices are made to preserve boundedness and conservation {\color{blue} of momentum and energy} when mass is transferred between fluids.

The formulation of transfer terms in order to represent sub-grid convection is the subject of future work.

%\keywords{Convection, atmpspheric, modelling, parameterisation, grey zone}
\end{abstract}
\maketitle

\section{Introduction}

The representation of sub-grid scale convection is arguably the weakest
aspect of weather and climate prediction models \citep[eg.][]{SLF+10,SAB+13,HPB+14}
and leads to poor predictions of weather and climate in the extratropics
\citep[eg. ][]{LCD+08} and the tropics \citep[Chapter 8,][]{ipcc41}.
The problem gets worse \textcolor{blue}{when modelling convection
in the grey zone}, where convection is partially resolved and so the
assumptions made by most convection schemes are particularly bad \citep[eg. ][]{GG05}.
Two specific assumptions are identified which we aim to avoid:
\begin{enumerate}
\item Net mass flux by convection; traditional mass flux (and other) convection
schemes assume that convection does not create a net transport of
mass in the vertical \citep[eg. ][]{GR90}. Instead mass is mixed
within each column.
\item Non-equilibrium dynamics; traditional convection schemes ignore effects
due to changes in time of the properties of convection \citep[eg. ][]{KF90}. 
\end{enumerate}
\textcolor{blue}{Once a convection scheme transports mass as well
as heat, moisture and momentum, all of the same equations of motion
should be solved both inside and outside the convective plumes so
it no longer makes sense to have a convection paramterisation as a
module to a dynamical core \textendash{} they should both be solving
the same equations and so the same model should be used for both.
Dynamical cores all have some form of implicit solution to control
divergence, either treating acoustic waves implicitly, at least in
the vertical direction \citep[eg. ][]{SK92}, so that they do not
lead to severe time-step restrictions, or some form of implicit technique
to ensure (psuedo)-incompressibility \citep[eg.][]{SBJ+89}. Once
convection schemes can transport mass, they will trigger acoustic
and gravity waves and so terms of the equations related to convective
mass transport must appear in whatever method that the dynamical core
uses to control divergence. If, for example, 20\% of a grid box contains
a convective plume, a model will become unstable if the mass transported
by that plume is treated explicitly and deposited at the top of the
plume in one time-step. This mass transport must be treated consistently
with the mass transport of the resolved flow.}

The conditional averaging (or filtering) process for convection was
described by \citet{TWV+18} and involves multiplying each equation
of motion by an indicator function and averaging over a volume (or
applying a different filter). This leads to equations which are similar
to those of a mass flux convection scheme but without the approximation
of zero net mass flux by convection. The conditionally averaged equations
also have transfer terms to transfer mass, momentum, heat and moisture
between the fluids. These terms have a similar role to the closures
for cloud base mass flux and convective entrainment and detrainment. 

There are schemes which account for aspects of non-equilibrium dynamics
\citep[eg. ][]{GG05,YP12,Par14} but fewer that allow net mass flux
due to convection, exceptions being \citet{GG05,KGB07,AW13}. \citet{KGB07,KB08}
extend a mass flux convection scheme to transport mass in the vertical
by creating a source term of the continuity equation due to sub-grid
scale convection. Their approach is not as general or consistent as
that proposed by \citet{TWV+18} and it is also not clear if their
numerical technique will be stable for moderate time-steps. Other
attempts to allow net mass flux by convection \citep[eg.][]{GG05,AW13}
have relied on statistical approximations to define the area fraction
associated with convection rather than on prognostic equations, as
laid out by \citet{TWV+18}. A significant advance is the extended
EDMF scheme \citep{TKP+18} which presents conditionally averaged
equations of motion with different fluids for the environmental and
convective plume, including transport equations for the plume area
fraction. \citet{TKP+18} combine conditional averaging and Reynolds
averaging, presenting transport equations for sub-grid scale variability
in each fluid. However the numerical solutions that they present are
in a single column and they assume that no net mass is transferred
out of the column in order to simplify their numerical solution. In
order to make full use of the extended EDMF scheme, a robust numerical
method is needed to solve conditionally averaged equations for convection
in three dimensions. 

Conditional averaging has been used in other fields for decades; \citet{Dopa77}
described how it could be used for representing intermittent turbulent
flows but it has more commonly been used to represent multiphase flow
\citep[eg. ][]{LB91,GBB+07} with separate fluids for different phases
which share a single pressure. The conditionally averaged Euler equations
with a single pressure and without transfers between the fluids are
in fact ill-posed \citep{SW84} and are usually regularlised by including
coupling between phases such as drag and other relaxation transfers.
Alternatively, \citet{HK84} regularised these equations using multiple
pressures for problems with surface tension. \citet{TEB1x} are also
working on a single column solution of conditionally averaged equations
and show that the incompressible conditionally averaged equations
are unstable without additional coupling between the fluids. 

This paper presents a stable numerical method for solving the conditionally
averaged equations in arbitrary dimensions and proposes transfer terms
that transfer resolved convection into the buoyant fluid and stable
air back into the stable fluid. These transfer terms are not designed
to be used to represent sub-grid scale convection as this would require
more information about sub-grid scale variability. Instead they are
designed to be large source terms that will act to challenge the stability
of the numerical method\textcolor{blue}{{} as they are too big to be
treated explicitly and they will introduce new extrema}. We use \textcolor{blue}{two}
techniques to regularise the multifluid equations; the first is with
drag between fluids and the second is with diffusion between the fluids,
\textcolor{blue}{similar to lateral entrainment and detrainment}. 

\section{The Conditionally Averaged Euler Equations}

Traditional mass flux convection schemes solve simplified equations
of motion with temperature, vertical velocity and moisture inside
convective plumes. This is therefore a form of conditional averaging
with variables averaged inside and outside plumes. However we can
take the process further and avoid some of the crude assumptions made
by mass flux schemes such as vanishing convective area fraction and
no net mass flux due to convection. The conditional averaging (or
filtering) process for convection was described by \citet{TWV+18}
and involves multiplying each equation of motion by an indicator function,
$I_{i}$, for a number of different conditions labelled by $i$. $I_{i}$
is one if \textcolor{blue}{the fluid is defined to be fluid $i$ }at
that location and zero otherwise. A filter (typically volume average)
is then applied and averages for each condition can be found over
each filter region. The volume fraction in fluid $i$ is defined to
be
\begin{equation}
\sigma_{i}=\widetilde{I_{i}}
\end{equation}
where the $\widetilde{\ }$ implies the application of the filter
(or volume average). Density, potential temperature and velocity can
then be defined for each fluid:
\begin{eqnarray}
\rho_{i} & = & \widetilde{I_{i}\rho}/\sigma_{i}\\
\theta_{i} & = & \widetilde{I_{i}\rho\theta}\bigg/(\sigma_{i}\rho_{i})\\
\mathbf{u}_{i} & = & \widetilde{I_{i}\rho\mathbf{u}}\bigg/(\sigma_{i}\rho_{i})
\end{eqnarray}
and averages over all fluids (denoted by overbar) are:
\begin{eqnarray}
1 & = & \sum_{i}\sigma_{i}\\
\overline{\rho} & = & \sum_{i}\sigma_{i}\rho_{i}\\
\overline{\rho\theta} & = & \sum_{i}\sigma_{i}\rho_{i}\theta_{i}\\
\overline{\rho\mathbf{u}_{i}} & = & \sum_{i}\sigma_{i}\rho_{i}\mathbf{u}_{i}.
\end{eqnarray}
\textcolor{blue}{As with Reynolds or Favre averaging, }non-linear
conditionally averaged terms \textcolor{blue}{such as $\widetilde{I_{i}\rho\mathbf{u}\theta}$
and $\widetilde{I_{i}\rho\mathbf{u}\mathbf{u}}$ are not equal to}
the products of conditionally averaged terms. The difference is expressed
as a sub-filter scale flux:
\begin{eqnarray}
\widetilde{I_{i}\rho\mathbf{u}\theta} & = & \sigma_{i}\rho_{i}\mathbf{u}_{i}\theta_{i}+\mathbf{F}_{\text{SF}}^{\theta_{i}}\\
\widetilde{I_{i}\rho\mathbf{u}\mathbf{u}} & = & \sigma_{i}\rho_{i}\mathbf{u}_{i}\mathbf{u}_{i}+\mathsf{F}_{\text{SF}}^{\mathbf{u}_{i}}.
\end{eqnarray}
The sub-filter scale fluxes could be due to turbulent motions within
each fluid. \textcolor{blue}{Parameterisations for sub-filter scale
fluxes are not considered in this paper. We are still able to find
stable solutions of the equations because the flow that we are solving
is very well resolved. }

The same averaging can be applied to pressure but we will assume that
pressure is uniform across all fluids. The time-scale for equilibration
of pressure across all fluids \textcolor{blue}{is} related to the
speed of sound so uniform pressure across fluids may be an adequate
approximation and also makes numerical solution practical. \textcolor{blue}{However
the plume and environment are not in static equilibrium so pressure
differences will not all be equilibrated. Variations in the dynamics
between the plume and the environment will lead to form drag which
is discussed in section \ref{subsec:drag}. }

Uniform pressure across fluids is assumed when using conditional averaging
for representing multi-phase flows \citep[eg.][]{GBB+07}. A single
pressure for compressible multiphase flows is known to lead to an
ill-posed problem \citep{SW84} but the equations can be regularised
with some kind of coupling between fluids which will be discussed
in section \ref{subsec:transfers}. \citet{TKP+18} do not assume
that the pressure is equal in each fluid but they do assume that density
is equal in both fluids, except where it influences buoyancy. \citet{TKP+18}
also assume that drag is high enough that horizontal velocities are
equal between fluids. 

Applying conditional averaging to the \textcolor{blue}{dry, adiabatic},
rotating, compressible Euler equations in flux form, assuming uniform
pressure between fluids and ignoring sub-fiter scale fluxes leads
to the following conditionally averaged Euler equations \textcolor{blue}{for
mass, potential temperature and momentum}:
\begin{eqnarray}
\frac{\partial\sigma_{i}\rho_{i}}{\partial t} & + & \nabla\cdot(\sigma_{i}\rho_{i}\mathbf{u}_{i})=\sum_{j\ne i}\left(\sigma_{j}\rho_{j}S_{ji}-\sigma_{i}\rho_{i}S_{ij}\right)\label{eq:condCont}\\
\frac{\partial\sigma_{i}\rho_{i}\theta_{i}}{\partial t} & + & \nabla\cdot\left(\sigma_{i}\rho_{i}\mathbf{u}_{i}\theta_{i}\right)\label{eq:condTheta}\\
 & = & \sum_{j\ne i}\left(\sigma_{j}\rho_{j}\theta_{j}S_{ji}-\sigma_{i}\rho_{i}\theta_{i}S_{ij}-\sigma_{i}\rho_{i}H_{ij}\right)\nonumber \\
\frac{\partial\sigma_{i}\rho_{i}\mathbf{u}_{i}}{\partial t} & + & \nabla\cdot\left(\sigma_{i}\rho_{i}\mathbf{u}_{i}\mathbf{u}_{i}\right)=-2\sigma_{i}\rho_{i}\boldsymbol{\Omega}\times\mathbf{u}_{i}\label{eq:condMom}\\
 & - & \sigma_{i}\rho_{i}c_{p}\theta_{i}\nabla\pi+\sigma_{i}\rho_{i}\mathbf{g}\\
 & + & \sum_{j\ne i}\left(\sigma_{j}\rho_{j}\mathbf{u}_{j}S_{ji}-\sigma_{i}\rho_{i}\mathbf{u}_{i}S_{ij}-\sigma_{i}\sigma_{j}\mathbf{d}_{ij}\right)
\end{eqnarray}
where $\pi=(p/p_{0})^{\kappa}$ is the Exner pressure, $p$ is the
pressure, $p_{0}$ is a reference pressure, $\kappa=R/c_{p}$, $R$
is the gas constant of dry air, $c_{p}$ is the heat capacity of dry
air at constant pressure, $\theta=T/\pi$ is the potential temperature,
$\boldsymbol{\Omega}$ is the rotation rate of the domain, $\mathbf{g}$
is the acceleration due to gravity, $\sigma_{i}\rho_{i}S_{ij}$ is
the transfer rate of mass from fluid $i$ to fluid $j$, $\mathbf{D}_{ij}$
is the drag exerted from fluid $i$ onto fluid $j$ and $H_{ij}$
is the heat transfer from fluids $i$ to $j$. When mass is transferred,
\textcolor{blue}{we assume that} its \textcolor{blue}{mean} properties
are taken with it which is why $S_{ij}$ appears in all equations.
We then also assume the equation of state for dry air both globally
and for each fluid is given by:
\begin{equation}
p_{0}\pi^{\frac{1-\kappa}{\kappa}}=R\rho_{i}\theta_{i}=R\overline{\rho\theta}=R\sum_{i}\sigma_{i}\rho_{i}\theta_{i}.\label{eq:condState}
\end{equation}
The \textcolor{blue}{temperature and momentum equations} can be expressed
in advective form so that the primitive variables $\mathbf{u}_{i}$
and $\theta_{i}$ are well defined when $\sigma_{i}$ is zero:
\begin{eqnarray}
\frac{\partial\theta_{i}}{\partial t} & + & \mathbf{u}_{i}\cdot\nabla\theta_{i}=\sum_{j\ne i}\left(\frac{\sigma_{j}\rho_{j}}{\sigma_{i}\rho_{i}}S_{ji}(\theta_{j}-\theta_{i})-H_{ij}\right)\label{eq:condThetaAdv}\\
\frac{\partial\mathbf{u}_{i}}{\partial t} & + & \mathbf{u}_{i}\cdot\nabla\mathbf{u}_{i}=-2\boldsymbol{\Omega}\times\mathbf{u}_{i}-c_{p}\theta_{i}\nabla\pi+\mathbf{g}\label{eq:condMomAdv}\\
 & + & \sum_{j\ne i}\left(\frac{\sigma_{j}\rho_{j}}{\sigma_{i}\rho_{i}}S_{ji}(\mathbf{u}_{j}-\mathbf{u}_{i})-\mathbf{D}_{ij}\right)
\end{eqnarray}
\textcolor{blue}{where $D_{ij}=\sigma_{j}d_{ij}/\rho_{i}$. }Note
that if $\sigma_{i}$ is zero, there is a division by zero in the
mass transfer terms in eqns (\ref{eq:condThetaAdv}) and (\ref{eq:condMomAdv})
which leads to an infinite source term when $\theta$ and $\mathbf{u}$
are transferred to an empty fluid. This is appropriate because when
mass is transferred to an empty fluid, the properties should instantaneously
become those of the transferred fluid; the old properties of the empty
fluid should not have any influence. However this infinite source
term will require careful numerical treatment. 

\subsection{Transfers and Exchanges between Fluids\label{subsec:transfers}}

The conditionally averaged equations can only become useful for representing
sub-grid scale convection when transfer terms $\mathbf{D}_{ij}$,
$S_{ij}$ and $H_{ij}$ are formulated. $S_{ij}$ is particularly
important for moving mass in and out of the fluid related to convection
and will inevitably be associated with sub-grid scale variability
of buoyancy and other properties related to convection. In this paper
we do not propose new closures for how mass moves from the stable
to the convectively active fluid (such as cloud base mass flux). Instead
we will use transfer terms formulated in terms of differential operators
that transfer resolved flow into the convectively active fluid and
back again. The purpose of this is to test the stability, boundedness
and conservation properties of the numerical methods rather than proposing
a useful parameterisation of convection.

\textcolor{blue}{We also include transfer terms to stabilise the equations.
For further justification of the need for stabilisation, we should
consider the physical meaning of a mixture of fluids. $0<\sigma_{0}<1$
implies that at least two fluids are present at scales down to the
grid-scale. This implies that there will be a large surface area between
the two fluids and so they are likely to exchange mass and momentum.
We therefore couple the two fluids using mass exchanges or drag between
the fluids. }

\textcolor{blue}{We will consider one form of drag and three types
of mass transfer. The total mass transfer from fluid $i$ to $j$
is:
\begin{equation}
S_{ij}=S_{dij}+S_{bij}+S_{hij}
\end{equation}
where $S_{dij}$ is due to diffusion of $\sigma$, $S_{bij}$ is due
to buoyancy perturbations and $S_{hij}$ is due to horizontal divergence.
$S_{bij}$ and $S_{hij}$ are forms of cloud base mass flux and can
be compared with the cloud base mass flux of \citet{TKP+18} which
occurs at the ground and is based on sub-grid-scale variability. $S_{dij}$
is similar to lateral entrainment and is used to regularise the ill-posed
equations. These and the drag term will be described next. }Throughout
we assume $H_{ij}=0$.

\subsubsection{Drag in the Momentum Equation\label{subsec:drag}}

Drag between fluids is \textcolor{blue}{used to parameterise the pressure
differences between the fluids which lead to form drag. \citet{DSJD12}
used LES data to show that this is a large sub-grid scale term. We
use a model for the drag on a rising bubble based on} \citet{RLD+11}.
Assuming exactly two fluids and remembering that we need $\sigma_{i}\rho_{i}D_{ij}=-\sigma_{j}\rho_{j}D_{ji}$
we can use: 
\begin{equation}
\mathbf{D}_{ij}=\frac{\sigma_{j}}{\rho_{i}}\frac{C_{D}\rho_{ij}}{r_{c}}|\mathbf{u}_{i}-\mathbf{u}_{j}|\left(\mathbf{u}_{i}-\mathbf{u}_{j}\right)\label{eq:dragBubble}
\end{equation}
where $C_{D}$ is a drag coefficient, $r_{c}$ is the radius or length
scale of a region of fluid (which needs to be the same for both fluids)
and \textcolor{blue}{$\rho_{ij}=(\sigma_{i}\rho_{i}+\sigma_{j}\rho_{j})/(\sigma_{i}+\sigma_{j})$}.
As $\sigma_{i}$ becomes small in any fluid, we need $r_{c}$ to become
small which increases the drag between fluids. If we assume a maximum
and minimum, $r_{\max}$ and $r_{\min}$, then $r_{c}$ can take the
form:
\begin{equation}
r_{c}=\max\left(r_{\min},\ \sigma_{i}\sigma_{j}r_{\max}\right).\label{eq:dragRadius}
\end{equation}
\textcolor{blue}{For a more realistic scheme, future work could draw
on existing parameterisations of cloud radius.}

\subsubsection{Diffusion of $\sigma$\label{subsec:diffuseSigma}}

\textcolor{blue}{A diffusive mass transfer term is created to smooth
out high curvature or oscillations in $\sigma$. The form used ensures
that total mass is not diffused and that the transfer from $i$ to
$j$ is always positive:}

\begin{equation}
\sigma_{i}\rho_{i}S_{dij}=\frac{K_{\sigma}}{2}\max\left(\nabla^{2}\left(\sigma_{j}\rho_{j}-\sigma_{i}\rho_{i}\right),\ 0\right)\label{eq:diffusionTransfer}
\end{equation}
where $K_{\sigma}$ is a diffusion coefficient which can be chosen
so as to obey stability constraints if the mass transfer term is treated
explicitly in equation (\ref{eq:condCont}). This mixing of the fluids
is similar to turbulent entrainment at the lateral edges of plumes
but eqn (\ref{eq:diffusionTransfer}) is different from entrainment
rates in other schemes which depend on buoyancy and vertical velocity
\citep[eg. ][]{TKP+18} or on plume radius \citep[eg. ][]{DWB+13}.
In future work, existing parameterisations for entrainment could be
used to set $K_{\sigma}$.

\subsubsection{Transfers based on buoyancy perturbations\label{subsec:buoyancyTransfer}}

\textcolor{blue}{This transfer term has only been derived considering
two fluids, the stable environment ($i=0$) and the buoyant plume
($i=1$). It is formulated to mimic a parameterisation for cloud base
mass flux rather than to regularise the equations: it will introduce
new extrema rather than removing extrema. Air will rise if buoyancy
perturbations make it lighter than the air above or lighter that the
surroundings. These observations are usually used in closures for
cloud base mass flux by testing the (conditional) stability of a parcel
\citep[eg. ][]{GR90}. }We want to formulate transfer terms as part
of PDEs rather than introducing criteria comparing the buoyancy of
grid boxes with surrounding grid boxes. Therefore we use the Laplacian
of $\theta$ to inform us of positive and negative perturbations.
There will be a positive perturbation if $\nabla^{2}\theta<0$ and
vice versa. :
\begin{eqnarray}
S_{b01} & = & \begin{cases}
-K_{\theta}\frac{\nabla^{2}\theta_{0}}{\theta_{0}} & \ \text{when}\ \nabla^{2}\theta_{0}<0\\
0 & \ \text{otherwise}
\end{cases}\label{eq:thetaTransfer01}\\
S_{b10} & = & \begin{cases}
K_{\theta}\frac{\nabla^{2}\theta_{1}}{\theta_{1}} & \ \text{when}\ \nabla^{2}\theta_{1}>0\\
0 & \ \text{otherwise}
\end{cases}\label{eq:thetaTransfer10}
\end{eqnarray}
where $K_{\theta}$ is a diffusivity. 

\subsubsection{Transfers Based on Horizontal Divergence and Vertical Velocity\label{subsec:hdiv_w_Transfer}}

\textcolor{blue}{This transfer term has also been derived considering
only two fluids, the stable environment ($i=0$) and the buoyant plume
($i=1$). It }is also formulated to mimic convection rather than to
regularise the equations. The transfer term moves fluid from fluids
zero to one when fluid zero is converging in the horizontal and rising
and that moves fluid from fluid one to fluid zero when fluid one is
diverging in the horizontal and falling:
\begin{eqnarray}
\sigma_{0}\rho_{0}S_{h01} & = & \begin{cases}
-\nabla_{h}\cdot(\sigma_{0}\rho_{0}\mathbf{u}_{0}) & \text{if}\ \nabla_{h}\cdot(\sigma_{0}\rho_{0}\mathbf{u}_{0})<0\\
 & \text{and}\ \mathbf{u}_{0}\cdot\mathbf{g}<0\\
0 & \ \text{otherwise}
\end{cases}\label{eq:divTransfer01}\\
\sigma_{1}\rho_{1}S_{h10} & = & \begin{cases}
\nabla_{h}\cdot(\sigma_{1}\rho_{1}\mathbf{u}_{1}) & \text{if}\ \nabla_{h}\cdot(\sigma_{1}\rho_{1}\mathbf{u}_{1})>0\\
 & \text{and}\ \mathbf{u}_{1}\cdot\mathbf{g}>0\\
0 & \ \text{otherwise.}
\end{cases}\label{eq:divTransfer10}
\end{eqnarray}
\textcolor{blue}{This transfer term is related to closure based on
moisture convergence.}

\section{Semi-Implicit Numerical Solution}

\textcolor{blue}{Solving the conditionally averaged equations needs
an entire atmospheric model rather than just the convection parameterisation.
Once the sub-grid-scale convection interacts with the continuity equation,
convection can no longer be an isolated parameterisation \textendash{}
the whole model needs to change, particularly if the mass changes
due to convection are to be treated implicitly, which is needed to
avoid time-step restrictions based on the acoustic or gravity wave
speed. Therefore the entire numerical model is described here. Aspects
that are not specific to the stability of solving the conditionally
averaged equations are given in appendix \ref{appx:modelSpace}. }

The equations are discretised and solved using the OpenFOAM library
(\url{https://openfoam.org}) and the full implementation is part
of the AtmosFOAM repository (\url{https://github.com/AtmosFOAM/}).
The spatial discretisation uses standard OpenFOAM operators. 

\subsection{Spatial Discretisation \label{subsec:modelSpace}}

\textcolor{blue}{Most of the spatial discretisation is not specific
to the stable solution of the conditionally averaged equations and
is described in appendix \ref{appx:modelSpace}. }The spatial discretisation
uses a finite-volume C-grid for an arbitrary mesh, similar to that
described by \citet{WS14} with $\theta_{i}$ , $\sigma_{i}\rho_{i}$
and $\pi$ defined as volumetric mean quantities (or at cell centres)
and normal components of velocity defined on cell faces. All the meshes
used are orthogonal and the focus of this paper is not spatial discretisation
therefore for simplicity the discretisation is described for orthogonal
meshes. 

\textcolor{blue}{Specific to the conditionally averaged equations,
it is important to use bounded advection of $\sigma_{i}$. A TVD advection
scheme with a van-Leer limiter is described in appendix \ref{subsec:vanLeerContinuity}.
Note that it is the advection of $\sigma_{i}$ that is bounded, not
the advection of $\sigma_{i}\rho_{i}$ because it is necessary to
discretise $\rho_{i}$ consistently with the pressure for both stability
and energy conservation. This is described in more detail in sections
\ref{subsec:solveContinuity} and \ref{subsec:consistency}. Bounded
advection of $\rho_{i}$ is not needed because density is sufficiently
smooth and far from zero so that it remains positive.}

\subsection{Time Stepping Algorithm}

\textcolor{blue}{It is crucial to get the time-stepping right for
the stable solution of the conditionally averaged equations. In this
section we will distinguish between convenient modelling choices and
choices that are specific to the stable solution of the conditionally
averaged equations. The solution of the conditionally averaged equations
is interwoven with a semi-implicit method so that acoustic waves are
treated implicitly for large and small $\sigma_{i}$.}

The terms of the Euler equations involved in acoustic waves are solved
using second-order Crank-Nicolson time-stepping. \textcolor{blue}{Other
implicit schemes could be chosen and optionally gravity waves could
also be treated implicitly. For the stable solution of the conditionally
averaged equations it is necessary to include updates from both the
buoyant and stable momentum and continuity equations in the implicit
solution of the pressure equation.}

\textcolor{blue}{Non-linear terms and advection are treated explicitly
using second-order Runge-Kutta time-stepping for convenience \citep{WLW13}.
An inner loop solves the pressure equation twice with explicit terms
updated for the second iteration. The inner iterations are indexed
with $\ell=1,2$. An outer loop solves the continuity and temperature
equations explicitly. The outer iterations are indexed with $k=1,2$.
This low Mach number semi-implicit method follows \citet{WS14} and
is similar to \citet{WSW+14} in the number of inner and outer loops.}

\textcolor{blue}{All transfer terms are solved using operator split
either explicit or implicit first-order time-stepping. The operator
splitting is a straightforward way to ensure positivity. Implicit
updates are necessary for stable treatment of the large source terms
(the source terms with $\sigma_{i}\rho_{i}$ in the denominator which
can tend to infinity as $\sigma_{i}\rightarrow0$).}

\subsubsection{Initialisation \label{subsec:initialisation}}

\textcolor{blue}{The prognostic variables are $U_{i}$, $\sigma_{i}\rho_{i}$,
$\theta_{i}$ where $U_{i}=\mathbf{u}_{i}\cdot\mathbf{S}_{f}$ is
the volume flux across each cell face ($\mathbf{S}_{f}$ is the vector
normal to each face with magnitude of the face area).} Transport equations
are solved for $U_{i}$, $\sigma_{i}\rho_{i}$, $\theta_{i}$ and
$\pi$. This system is over specified because $\pi$ can be calculated
from all of the $\sigma_{i}\rho_{i}$ and $\theta_{i}$ using the
equation of state. To avoid over specification, only $\mathbf{u}_{i}$,
$\sigma_{i}$, $\theta_{i}$ and $\pi$ are read in at initialisation
\textcolor{blue}{and each $\rho_{i}$ is calculated from the equation
of state (\ref{eq:condState}). A separate code for calculating the
initial conditions calculates $\pi$ that is in discrete hydrostatic
balance with $\theta_{0}$.}

\textcolor{blue}{In order to start each time-step, $(\sigma_{i}\rho_{i})^{k=0}$
is set to $(\sigma_{i}\rho_{i})^{n}$ where $n$ is the label for
the old time step, $\theta_{i}^{k=0}$ is set to $\theta_{i}^{n}$
, $\pi^{\ell=0}$ is set to $\pi^{n}$ and $U_{i}^{\ell=0}$ to $U_{i}^{n}$.
After the loops over $k$ and $\ell$, the values at time $n+1$ are
set the final values at the end of the iterations. }

\subsubsection{Solving the Continuity Equation\label{subsec:solveContinuity}}

\textcolor{blue}{The first equations to be solved in the outer loop
are the continuity equations for each $\sigma_{i}\rho_{i}$. When
transfer terms are included (which lead to terms with $\sigma_{i}\rho_{i}$
in the denominator) it is essential to keep $\sigma_{i}\rho_{i}$
positive for stability. The continuity equations are solved using
operator splitting, first advecting $\sigma_{i}\rho_{i}$ and then
applying the transfer terms. This is the most straightforward way
to maintain positivity. We have chosen }a TVD advection scheme with
a van-Leer limiter (section \ref{subsec:vanLeerContinuity}) \textcolor{blue}{to
advect } \textcolor{blue}{$\sigma_{i}\rho_{i}$ for iterations $k=1,2$:
\begin{equation}
(\sigma_{i}\rho_{i})^{\prime}=(\sigma_{i}\rho_{i})^{n}-\Delta t\ \nabla\cdot\left(\left[(1-\alpha)\rho_{i}^{n}\mathbf{u}_{i}^{n}+\alpha\rho_{i}^{k-1}\mathbf{u}_{i}^{\ell-1}\right]\sigma_{i}^{n}\right)\label{eq:advectSigma}
\end{equation}
where $\Delta t$ is the time-step and $\alpha$ is the off-centering
parameter. For all the simulations presented, $\alpha=1/2$ is used
making the time-stepping second-order accurate. Values of $\sigma_{i}\rho_{i}$
for all values of $k$ and for the intermediate values, $(\sigma_{i}\rho_{i})^{\prime}$
share the same memory as $(\sigma_{i}\rho_{i})^{n+1}$.}

\textcolor{blue}{Only $\sigma_{i}^{n}$ appears on the right hand
side of eqn (\ref{eq:advectSigma}), no newer values such as $\sigma_{i}^{k-1}$.
This is because the advection scheme assumes that the upwind values
are at the old time-level and it is only guaranteed bounded when using
$\sigma_{i}^{n}$ on the right hand side. However updated values of
$\rho_{i}$ are used on the right hand side of eqn (\ref{eq:advectSigma})
so that the solution of eqn (\ref{eq:advectSigma}) remains consistent
with the Exner pressure, $\pi$.}

\textcolor{blue}{Next the mass transfer terms are calculated using
$\sigma_{i}^{\prime}$, $\mathbf{u}_{i}^{\ell-1}$ and $\theta_{i}^{\ell-1}$
(the most up to date values) and limited to ensure that $\sigma_{i}\rho_{i}$
remains positive:
\begin{equation}
S_{ij}^{\text{lim}}=\frac{1}{(\sigma_{i}\rho_{i})^{\prime}}\min\left((\sigma_{i}\rho_{i})^{\prime}S_{ij},\ \frac{(\sigma_{i}\rho_{i})^{\prime}-\sigma_{\min}\rho_{i}^{k-1}}{\Delta t}\right)
\end{equation}
}where $\sigma_{\min}=10^{-9}$ is used in the simulations presented
in section \ref{sec:results}. Then the mass transfer is used to update
$\sigma_{i}\rho_{i}$ explicitly with operator splitting:\textcolor{blue}{
\begin{equation}
(\sigma_{i}\rho_{i})^{k}=(\sigma_{i}\rho_{i})^{\prime}+\Delta t\left((\sigma_{j}\rho_{j})^{\prime}S_{ji}^{\text{lim}}-(\sigma_{i}\rho_{i})^{\prime}S_{ij}^{\text{lim}}\right).\label{eq:transferMass}
\end{equation}
}

\subsubsection{Solving the $\theta_{i}$ equation\label{subsec:solveTheta}}

After the continuity equation, the $\theta_{i}$ equation is solved
using operator splitting; first advecting $\theta_{i}$ then applying
the mass transfer terms to the advected $\theta_{i}$. The mass transfer
terms are applied implicitly because they can be very large due to
\textcolor{blue}{the presence of $\sigma_{i}\rho_{i}$ in the denominator}.
Because we are using a finite volume model to solve equations in advective
form, the advection of $\theta_{i}$ is calculated as:
\begin{multline}
\theta_{i}^{\prime}=\theta_{i}^{n}-\Delta t\ \biggl((1-\alpha)\left[\nabla\cdot(\theta_{i}\mathbf{u}_{i})-\theta_{i}\nabla\cdot\mathbf{u}_{i}\right]^{n}\\
+\alpha\left[\nabla\cdot(\theta_{i}^{k-1}\mathbf{u}_{i}^{\ell-1})-\theta_{i}^{k-1}\nabla\cdot\mathbf{u}_{i}^{\ell-1}\right]\biggr)\label{eq:thetaAdvectDt}
\end{multline}
where the spatial discretisation is described in section \ref{subsec:thetaAdvect}.

The implicit addition of mass transfer terms is formulated to be specific
for having two fluids although it would be straightforward to generalise.
In order to derive the equations for adding the mass transfer terms
to $\theta_{i}$ we will write the $\theta_{i}$ equation as:
\begin{equation}
\theta_{i}^{k}=\theta_{i}^{\prime}+\Delta t\sum_{j\ne i}\left(\frac{(\sigma_{j}\rho_{j})^{\prime}}{(\sigma_{i}\rho_{i})^{\prime}}S_{ji}^{\text{lim}}(\theta_{j}^{k}-\theta_{i}^{k})\right).\label{eq:thetaAddSource}
\end{equation}
Note values at level $k$ are on the left and right hand side making
this is an implicit solution. For $i=0,1$ this can be re-arranged
to give:
\begin{eqnarray}
\theta_{0}^{k} & = & \frac{\left(1+\Delta t\ T_{01}\right)\theta_{0}^{\prime}+\Delta t\ T_{10}\theta_{1}^{\prime}}{1+\Delta t\ T_{10}+\Delta t\ T_{01}}\label{eq:theta0Transfer}\\
\theta_{1}^{k} & = & \frac{\theta_{1}^{\prime}+\Delta t\ T_{01}\theta_{0}^{k}}{1+\Delta t\ T_{01}}\label{eq:theta1Transfer}
\end{eqnarray}
where $T_{ij}=\frac{(\sigma_{i}\rho_{i})^{\prime}}{(\sigma_{j}\rho_{j})^{\prime}}S_{ij}^{\text{lim}}$
is calculated just before eqn (\ref{eq:transferMass}) so as to use
$(\sigma_{i}\rho_{i})^{\prime}$ before it is over-written by $(\sigma_{i}\rho_{i})^{k}$.
For conservation of internal energy, it is necessary that the values
of $\sigma_{i}\rho_{i}$ from after advection but before the mass
transfer are used in the calculation of $T_{ij}$. This ensures that
$\sum_{\text{cells}}\sum_{i}(\sigma_{i}\rho_{i}\theta_{i})^{n}=\sum_{\text{cells}}\sum_{i}(\sigma_{i}\rho_{i}\theta_{i})^{n+1}$.
\textcolor{blue}{Calculation of $\theta_{i}^{k}$ from eqns (\ref{eq:theta0Transfer})
and (\ref{eq:theta1Transfer})} also ensures boundedness of $\theta_{i}$
($\theta_{i}^{k}$ will remain between $\theta_{i}^{k-1}$ and $\theta_{j}^{k-1}$).

\subsubsection{Diagnosing $\sigma_{i}$\label{subsec:diagnoseSigma}}

After the updates of prognostic variables $(\sigma_{i}\rho_{i})$
and $\theta_{i}$, the diagnostic variable $\sigma_{i}$ can be updated.
$\sigma_{i}$ is not used in isolation from $\rho_{i}$ anywhere in
the equations (\ref{eq:condCont}, \ref{eq:condState}-\ref{eq:condMomAdv}).
However $\sigma_{i}$ and $\rho_{i}$ may be needed independently
in closure assumptions, such as the approximation of the drag (below).
Firstly, each $\rho_{i}$ is calculated from the equation of state
using the most up to date values:
\begin{equation}
\rho_{i}^{k}=\frac{p_{0}\pi^{\frac{1-\kappa}{\kappa}}}{R\theta_{i}}=\frac{\overline{\rho\theta}}{\theta_{i}}.
\end{equation}
Then each $\sigma_{i}$ can be calculated:
\begin{equation}
\sigma_{i}^{k}=\frac{(\sigma_{i}\rho_{i})^{k}}{\rho_{i}^{k}}.\label{eq:diagnoseSigma}
\end{equation}
This calculation will guarantee $\sum_{i}\sigma_{i}=1$.

\subsubsection{Momentum and Continuity\label{subsec:Helmholtz}}

\textcolor{blue}{Here we describe how a standard semi-implicit algorithm
is adapted to solve the momentum and continuity equations from all
fluids simultaneously. }The momentum and continuity equations are
combined to form a Helmholtz equation for $\pi$. This is done by
expressing the volume flux, $U_{i}$, and the mean density, $\overline{\rho}$,
as linear functions of $\pi$ and substituting these into the continuity
equation. \textcolor{blue}{This is done twice per outer loop, in an
inner loop indexed by $\ell$.}

The normal component of the volume flux, $U_{i}$ is expressed as
a linear function of $\pi$ using the momentum equation:
\begin{equation}
U_{i}^{\prime}=U_{i}^{\prime\prime}-\alpha\Delta tc_{p}\theta_{fi}^{\ell}\nabla_{S}\pi^{\ell}\label{eq:volFluxFromMom}
\end{equation}
where $c_{p}\theta_{fi}^{\ell}\nabla_{S}\pi^{n+1}$ is an approximation
of $c_{p}\theta_{i}\nabla\pi\cdot\mathbf{S}_{f}$ and is calculated
from equation (\ref{eq:gradP}) in appendix \ref{subsec:gradP}. $U_{i}^{\prime\prime}$
is the explicitly calculated part of the volume flux:
\begin{eqnarray}
U_{i}^{\prime\prime} & = & U_{i}^{n}\label{eq:Uprime}\\
 & - & (1-\alpha)\Delta t\left(\left[\mathbf{u}_{i}\cdot\nabla\mathbf{u}_{i}+2\boldsymbol{\Omega}\times\mathbf{u}_{i}\right]\cdot\mathbf{S}_{f}+c_{p}\theta_{fi}\nabla_{S}\pi\right)^{n}\nonumber \\
 & - & \alpha\Delta t\left(\mathbf{u}_{i}\cdot\nabla\mathbf{u}_{i}+2\boldsymbol{\Omega}\times\mathbf{u}_{i}\right)^{\ell-1}\cdot\mathbf{S}_{f}+\Delta t\mathbf{g}\cdot\mathbf{S}_{f}.\nonumber 
\end{eqnarray}
Equation (\ref{eq:volFluxFromMom}) is multiplied by the linear interpolate
of $\sigma_{i}\rho_{i}$ onto faces and then the sum is taken over
all fluids to get the total mass flux:
\begin{equation}
F^{\ell}=\sum_{i}(\sigma_{i}\rho_{i})_{f}^{k}U_{i}^{\prime\prime}-\alpha\Delta tc_{p}\overline{\rho\theta}_{f}^{k}\nabla_{S}\pi^{\ell}.\label{eq:fluxFromMom}
\end{equation}
This will be substituted into the divergence term of the continuity
equation once we have described the linear representation of $\rho$
as a function of $\pi$ which follows \citet{WS14}.

In order to derive a Helmholtz equation for $\pi$ using the continuity
equation, the density is expressed as
\begin{equation}
\overline{\rho}=\Psi\pi\label{eq:usePsi}
\end{equation}
where $\Psi$ is the compressibility from the equation of state:
\begin{equation}
\Psi=\overline{\rho}^{\frac{2\kappa-1}{\kappa-1}}\left(\frac{R\overline{\theta}}{p_{0}}\right)^{\frac{\kappa}{\kappa-1}}.\label{eq:Psi}
\end{equation}
Equations (\ref{eq:fluxFromMom}) and (\ref{eq:usePsi}) are substituted
into the continuity equation and Gauss's divergence theorem is used
to calculate the divergence term:
\begin{multline}
\frac{\Psi^{\ell-1}\pi^{\ell}-\Psi^{n}\pi^{n}}{\Delta t}+\frac{1-\alpha}{V}\sum_{f\in c}F^{n}\\
\frac{\alpha}{V}\sum_{f\in c}\left\{ \sum_{i}(\sigma_{i}\rho_{i})_{f}^{k}U_{i}^{\prime\prime}-\alpha\Delta tc_{p}\left(\overline{\rho\theta}_{f}^{k}\right)\nabla_{S}\pi^{\ell}\right\} =0\label{eq:Helmholtz}
\end{multline}
where $V$ is the cell volume. There are no source terms to this equation
because the source terms transfer mass between fluids and the total
continuity equation is summed over fluids. This is a Helmholtz equation
that can be solved for $\pi^{\ell}$. Back substitutions are then
made to calculate each $U_{i}^{\prime}$ using equation (\ref{eq:volFluxFromMom}). 

\paragraph{Applying Drag and Mass Transfer to the Momentum Equation}

The transfer terms of equation (\ref{eq:condMomAdv}) can be applied
after the solution of the Helmholtz equation because they do not directly
influence the pressure. They are applied implicitly, first-order with
operator splitting with a simultaneous solution for two fluids, $i$
and $j$. $U_{i}^{\prime}$ is the volume flux predicted by the back
substitution after the Helmholtz equation and $U_{i}^{\ell}$ is the
solution of $U_{i}$ after implicit application of the source terms:
\begin{eqnarray}
U_{i}^{\ell} & = & U_{i}^{\prime}-\Delta t\ \left(T_{ji}+\sigma_{j}\frac{C_{D}\overline{\rho}^{k}}{r_{c}\rho_{i}^{k}}|\mathbf{u}_{i}-\mathbf{u}_{j}|^{\ell-1}\right)U_{i}^{\ell}\nonumber \\
 & + & \Delta t\left(T_{ji}+\sigma_{j}\frac{C_{D}\overline{\rho}^{k}}{r_{c}\rho_{i}^{k}}|\mathbf{u}_{i}-\mathbf{u}_{j}|^{\ell-1}\right)U_{j}^{\ell}
\end{eqnarray}
Using the same block implicit solution technique as was described
in section \ref{subsec:solveTheta}, $U_{i}^{\ell}$ can be calculated
using:\textcolor{blue}{
\begin{eqnarray}
U_{0}^{\ell} & = & \frac{\left(1+\Delta t\ \mathbb{T}_{01}\right)U_{0}^{\prime}+\Delta t\ \mathbb{T}_{10}U_{1}^{\prime}}{1+\Delta t\ \mathbb{T}_{10}+\Delta t\ \mathbb{T}_{01}}\label{eq:Utransfer0}\\
U_{1}^{\ell} & = & \frac{U_{1}^{\prime}+\Delta t\ \mathbb{T}_{01}U_{0}^{\ell}}{1+\Delta\mathbb{T}_{01}}\label{eq:Utransfer1}
\end{eqnarray}
where $\mathbb{T}_{ij}=T_{ij}+\frac{\sigma_{i}^{k}}{\rho_{j}^{k}}\frac{C_{D}\overline{\rho}^{k}}{r_{c}}|\mathbf{u}_{j}-\mathbf{u}_{i}|^{\ell-1}$.
}As with the numerical method for applying the mass transfer terms
to the $\theta_{i}$ equations, this technique ensures that the $U_{i}$
remain bounded and the use of the values of $\sigma_{i}\rho_{i}$
from before mass transfer in the calculation of $T_{ij}$ gives momentum
conservation on transfer.

\subsubsection{\textcolor{blue}{Updates for Consistency \label{subsec:consistency}}}

\textcolor{blue}{So far, each time-step, we have updated each $\sigma_{i}\rho_{i}$
using the continuity equations, $U_{i}$ using the momentum equations,
$\theta_{i}$ using the temperature equations and and $\pi$ using
the Helmholz equation which is a combination of the momentum and continuity
equations and the equation of state. This is over-specified so to
avoid inconsistencies growing (which leads to instability), $\pi$
is re-calculated from $\sigma_{i}\rho_{i}$ and $\theta_{i}$ at the
end of every time-step using just the equation of state (\ref{eq:condState}).
The resulting changes to $\pi$ need to be small otherwise the updated
$\pi$ would not solve the Helmholtz equation and acoustic modes would
grow. Therefore $\rho_{i}$ must be solved with the same spatial discretisation
as $\pi$ which, for best energy conservation (given 2nd order numerics)
is centred linear differencing. This is why only $\sigma_{i}$, not
$\sigma_{i}\rho_{i}$ is solved with a bounded advection scheme as
described in section \ref{subsec:solveContinuity}. }

\subsubsection{Overview of the Solution Algorithm}

\textcolor{blue}{Each item below is carried out for all cells before
moving on to the next item. This is crucial for the semi-implicit
formulation. }
\begin{enumerate}
\item \textcolor{blue}{We start by initialising all iterated variables at
$k=0$ and $\ell=0$ to the values at time level $n$.}
\item For $k=1,2$:

\begin{enumerate}
\item Solve for $(\sigma_{i}\rho_{i})^{k}$ as described in section \ref{subsec:solveContinuity}.
\item Solve for $\theta_{i}^{k}$ as described in section \ref{subsec:solveTheta}.
\item Update $\sigma_{i}^{k}$ as described in section \ref{subsec:diagnoseSigma}.
\item \textcolor{blue}{For $\ell=1,2$}

\begin{enumerate}
\item Update each $U_{i}^{\prime\prime}$ using eqn (\ref{eq:Uprime}) which
consists of all of the terms of the momentum equation excluding the
pressure gradient term and excluding transfer terms.
\item Calculate the compressibility, $\Psi$, from eqn (\ref{eq:Psi}).
\item Construct and solve the Helmholtz eqn (\ref{eq:Helmholtz}) for $\pi^{\ell}$.
\item Back substitute, adding the pressure gradient term to $U_{i}^{\prime\prime}$
to get $U_{i}^{\prime}$ using eqn (\ref{eq:volFluxFromMom}).
\item Add the transfer terms to $U_{i}^{\prime}$ to get $U_{i}^{\ell}$
using eqn (\ref{eq:Utransfer0},\ref{eq:Utransfer1}).
\item Calculate $\mathbf{u}_{i}^{\ell}$ from $U_{i}^{\ell}$ using eqn
(\ref{eq:recontructU}).
\end{enumerate}
\end{enumerate}
\item For consistency, update $\Psi$ and $\pi$ at the end of each time-step
\textcolor{blue}{from the equation of state} (section \ref{subsec:consistency}).
\end{enumerate}

\section{Results\label{sec:results}}

No test cases exist for numerical solutions of the conditionally averaged
Euler equations and so variations of the rising bubble test case \citep{BF02}
for the non-hydrostatic, compressible Euler equations are used. If
the conditions in each fluid are initially identical, then the solution
should evolve exactly like the single fluid equations with an additional
advected tracer for the fluid fraction. This is therefore used as
a first test of the numerical method. Tests are next formulated with
different initial conditions in each fluid in order to check that
the solution maintains stability, boundedness and some conservation
properties. Finally, tests are created with fluid one initially empty
and mass is transferred in. The solution should evolve exactly as
the single fluid case because the initial conditions in fluid one,
with no mass, should be irrelevant. 

The dry, warm rising bubble test case of \citet{BF02} consists of
a two dimensional vertical slice of height 10\,km and width 20\,km
initially at rest with a surface pressure of 1000\,mb, an initially
uniform potential temperature of 300\ K. The initial pressure is
in discrete hydrostatic balance with this uniform potential temperature.
A warm perturbation: 
\begin{equation}
\theta^{\prime}=2\cos^{2}\frac{\pi L}{2}\label{eq:thetaPerturb}
\end{equation}
is added for $L<1$ where $L=\sqrt{\left(\frac{x-x_{c}}{x_{r}}\right)^{2}+\left(\frac{z-z_{c}}{z_{r}}\right)^{2}}$,
$x_{c}=10\ \text{km}$, $z_{c}=2\ \text{km}$ and $x_{r}=z_{r}=2\ \text{km}$.
100\,m grid spacing is used and for all simulations presented a time-step
of 2\,s is used. Regardless of the initial conditions and transfers
between fluids, $\sigma$ should remain bounded between zero and one
and the potential temperature should remain bounded between $300\ $K
and $302\ $K.

\subsection{Two Identical Fluids}

First the warm rising bubble of \citet{BF02} is simulated with the
fluid divided into two fluids with identical initial conditions in
each fluid. No transfers or exchanges between fluids are used. Two
different initial fluid fractions are used as shown at the top of
figure \ref{fig:identicalParts}. These should not influence the evolution
of other variables. The two initial $\sigma$ distributions are
\begin{eqnarray}
\text{symmetric:}\ \sigma & = & \begin{cases}
1 & \ \text{if}\ \biggl|\mathbf{x}-\left(\begin{array}{c}
0\\
2
\end{array}\right)\text{km}\biggr|<2\ \text{km}\\
0 & \ \text{otherwise},
\end{cases}\\
\text{asymmetric:}\ \sigma & = & \begin{cases}
1 & \ \text{if}\ \biggl|\mathbf{x}-\left(\begin{array}{c}
2\\
5
\end{array}\right)\text{km}\biggr|<2\ \text{km}\\
0 & \ \text{otherwise}.
\end{cases}
\end{eqnarray}
The distributions of $\sigma$, $\theta_{i}$ and the velocity in
each fluid after 1000\,s are shown at the bottom of figure \ref{fig:identicalParts}.
$\theta$ and the velocity have remained identical in each fluid and
$\sigma$ has been advected by the flow without any undershoots or
overshoots. The presence of the $\sigma$ field does not influence
the evolution of the velocity or potential temperature in each fluid,
as expected. 

\begin{figure*}
\includegraphics[width=1\linewidth]{figures/identicalParts}

\caption{Initial fluid fraction and $\theta_{i}$ (top) for simulations with
identical properties in each fluid and properties after 1000\,s (bottom).
The vectors and contours for $\theta_{i}$ and $\mathbf{u}_{i}$ for
each fluid are identical. \textcolor{blue}{The full domain extends
between $x=-10^{3}\ $m and $x=10^{3}\ $m.} \label{fig:identicalParts}}
\end{figure*}

The stability of the model is demonstrated by plotting the total energy
and the different types of energy in figure \ref{fig:energy1part}.
The left hand side shows normalised kinetic, potential, internal and
total energy changes for the model with a single fluid using van-Leer
advection. The various energies are defined as:
\begin{eqnarray}
\text{KE} & = & \frac{1}{2}\sum_{i}\sigma_{i}\rho_{i}|\mathbf{u}_{ci}|^{2}\\
\text{PE} & = & -\mathbf{g}\cdot\mathbf{x}\sum_{i}\sigma_{i}\rho_{i}\\
\text{IE} & = & c_{v}\pi\sum_{i}\sigma_{i}\rho_{i}\theta_{i}\\
E & = & \text{KE}+\text{PE}+\text{IE}
\end{eqnarray}
and totals are calculated by integrating over space. The normalisation
and calculation of changes is calculated for energy XE as:
\begin{equation}
\widetilde{\text{\text{XE}}}=\frac{\text{\text{XE}}-\text{\text{XE}}(t=0)}{E(t=0)}.
\end{equation}
The dashed lines in figure \ref{fig:energy1part} show negative values
and the solid lines show positive values. In the first part of the
simulation, the single fluid simulation shows internal and potential
energy oscillating in phase with each other, showing nearly conservative
transfers between internal and potential energy. Throughout the simulation
the kinetic energy increases as the rising bubble accelerates while
the total energy decreases monotonically due to stable, dissipative
nature of the model. Part of this dissipation is due to the dissipative
advection of velocity \textcolor{blue}{(linear-upwind)} and potential
temperature \textcolor{blue}{(van-Leer)}. A simulation is also run
using centred, linear differencing for advection and the total energy
is shown in the right hand of figure \ref{fig:energy1part}. This
simulation looses energy less slowly and the energy loss is no longer
monotonic. The results of this simulation are noisy but stable with
overshoots and undershoots of temperature (not shown).

The accuracy of the energy conservation in figure \ref{fig:energy1part}
appears to be good partly because the energy changes are divided by
a large number \textendash{} the total initial energy, including unavailable
energy. The initial potential energy, which is mostly unavailable,
is 31,444\ Joules and the initial internal energy is 141,446\ Joules,
making the total initial energy 172,891\ Joules. A fairer accuracy
estimate might be to normalise with the available potential energy.
This has not been calculated. Instead we could compare with the energy
of the initial warm bubble. The warm bubble contains 24.5\ Joules
of additional internal energy in comparison to the stably stratified
state. If we were to normalise the energy changes with this value
then they would be 7,057 times bigger, making a normalised change
of $10^{-6}$ close to a change of 0.01.

The total energy for the the simulations with two identical fluids
with symmetric and asymmetric distributions of $\sigma$ are shown
in figure \ref{fig:energy1part}. This confirms that the presence
of more than one identical fluid does not influence the energy conservation.
In fact the solutions with two identical fluids are identical to the
solution with one fluid to within machine precision. 

\begin{figure*}
\includegraphics[width=1\linewidth]{figures/energy1part}

\caption{Normalised changes in kinetic, potential, internal and total energy
for the rising bubble test case for the model with a single fluid
and the normalised total energy change for models with one fluid with
different advection schemes and for a model with two identical fluids
and no mass transfer. Solid lines show positive changes and dashed
lines show negative changes.\label{fig:energy1part}}
\end{figure*}

\subsection{Different Initial Conditions in each Fluid}

Once each fluid has different properties, the behaviour of the solutions
changes and stable solutions may not exist. The total solution is
close to divergence free because compressibility is small in this
low Mach number regime. However with only one pressure to control
the divergence in two fluids, the divergence in each fluid can be
large. We will therefore initialise the model to force different velocities
and hence divergence in each fluid. We do not have an analytic solution
for this case but we seek stable solutions and we test energy conservation
since energy is conserved in the continuous equations in the absence
of transfers between the fluids.

In order to simulate two fluids with different properties occupying
the same location, we set $\sigma_{1}=\frac{1}{2}$ in a circle with
warm air only in fluid 1 and initially stationary flow in each fluid:
\begin{eqnarray}
\sigma_{1} & = & \begin{cases}
\frac{1}{2} & \ \text{if}\ \biggl|\mathbf{x}-\left(\begin{array}{c}
0\\
2
\end{array}\right)\text{km}\biggr|<2\ \text{km}\\
0 & \ \text{otherwise},
\end{cases}\\
\sigma_{0} & = & 1-\sigma_{1}\\
\theta_{0} & = & 300\ \text{K}\\
\theta_{1} & = & 300\ \text{K}\ +\theta^{\prime}.
\end{eqnarray}
We assume no mass flux and no drag between fluids. The initial conditions
for $\sigma_{1}$ and $\theta_{1}$ are shown in figure \ref{fig:diffuse1_noDrag}
along with the solutions after 100, 200 and 290\ s. $\theta_{0}$
remains identically 300\,K throughout the simulation, as expected.

\begin{figure*}
\includegraphics[width=1\linewidth]{figures/diffuse1_noDrag}

\caption{Zoomed in initial conditions and results of a simulation with a buoyant
perturbation in fluid 1 and no mass transfer. $\sigma_{1}$ is shaded,
$\theta_{1}$ is contoured every 0.2\,K, $\mathbf{u}_{0}$ is shown
by black vectors and $\mathbf{u}_{1}$ by red vectors. \textcolor{blue}{The
full domain extends between $x=-10^{3}\ $m and $x=10^{3}\ $m and
between $z=0$ and $z=10^{3}\ $m.} \label{fig:diffuse1_noDrag}}
\end{figure*}

The buoyancy perturbation in fluid one makes fluid one rise which
raises the pressure above the bubble which forces fluid zero downwards.
Consequently total divergence is controlled but the divergence in
each fluid grows and hence the velocities become large. The advection
scheme is only bounded for Courant numbers less than 0.5. At 276\,s,
the mean Courant number becomes larger than 0.5 and oscillations grow
in $\sigma_{1}$. The solution diverges at $t=296\ \text{s}$. A stable
simulation could be maintained for \textcolor{blue}{a little }longer
by using a smaller time-step or by treating advection implicitly \textcolor{blue}{but
the PDEs that we are solving are unstable \citep{SW84} so numerics
cannot be used to maintain stability indefinitely.} 

We first add drag between the fluids but no mass transfer. The drag
takes the form described in section \ref{subsec:drag} with length
scales $r_{\min}=1\ \text{m}$ (a minimum is needed to avoid division
by zero) and $r_{\max}=2000\ \text{m}$. A large value of $r_{\max}$
has been chosen so that the drag is low where neither $\sigma$ is
vanishing which allows some variation of velocity between fluids.
The results at $t=1000\ \text{s}$ are shown in figure \ref{fig:diffuse1_dragDiffuse}.
$\theta_{0}$ remains identically 300\,K throughout. The drag has
stabilised the solution and the two fluid velocities (in black and
red) are not identical. 

\begin{figure*}
\includegraphics[width=1\linewidth]{figures/diffuse1_dragDiffuse}

\caption{Results at $t=1000\ \text{s}$ of simulations with stabilisation with
a buoyant perturbation in fluid 1. $\sigma_{1}$ is shaded, $\theta_{1}$
and $\theta_{2}$ are contoured every 0.2\,K, $\mathbf{u}_{0}$ is
shown by black vectors and $\mathbf{u}_{1}$ by red vectors. Initially
$\sigma=\frac{1}{2}$ in circle near the ground (as in fig \ref{fig:diffuse1_noDrag}).
The time-series shows the normalised energy changes. Dashed lines
show negative changes. \label{fig:diffuse1_dragDiffuse}}
\end{figure*}

Next the fluids are coupled by adding diffusion \textcolor{blue}{between
the fluids (but no drag)} as described in section \ref{subsec:diffuseSigma}.
This is similar to convective entrainment and detrainment due to turbulence.
The results using a diffusion coefficient of $K_{\sigma}=200\ \text{m}^{2}\text{s}^{-1}$
are shown in figure \ref{fig:diffuse1_dragDiffuse}. The diffusion
is treated explicitly and this diffusion coefficient is well below
the stability limit. Once mass is transferred, \textcolor{blue}{the
peaks in $\sigma_{1}$ reduce in comparison to the simulation with
drag between the fluids. }Temperature and momentum are transferred
\textcolor{blue}{with the mass} so the temperature in fluid 1 no longer
remains 300\,K \textcolor{blue}{and the temperature in fluid 0 reduces}.

The total energy changes for all of the simulations with warm air
in a diffuse fluid 1 are shown in the bottom right of figure \ref{fig:diffuse1_dragDiffuse}.
The energy diverges for the unstable case with no transfers. The other
simulations are stabilised either by mass transfers or by momentum
transfer (drag). Both of these destroy kinetic energy and so we expect
to see the energy decrease monotonically for the stabilised simulations.
The simulation with the low diffusion coefficient ($K_{\sigma}=200\ \text{m}^{2}\text{s}^{-1}$)
stabilises the model after around 350\,s but this looks unreliable
as energy increases before this point. A larger diffusion coefficient,
($K_{\sigma}=800\ \text{m}^{2}\text{s}^{-1}$) stabilises the model
more effectively. (The stability limit is $K_{\sigma}\Delta t/\Delta x^{2}<\frac{1}{2}$
so for $\Delta t=2\text{s}$ and $\Delta x=100\text{m}$ the stability
limit is $K_{\sigma}=2500\ \text{m}^{2}\text{s}^{-1}$.) The simulation
is also stabilised by using drag between the fluids. The low drag
coefficient of $C_{D}=1$, $r_{\min}=1\ \text{m}$ and $r_{\max}=2000\ \text{m}$
leads more rapid energy loss than any of the other effective stabilisation
methods. A high drag coefficient ($C_{D}=10^{6}$) is also used to
check the stability of the implicit treatment of drag. The simulation
stabilised by mass transfer based on divergence looses energy monotonically,
more slowly than the other stabilisations. 

\subsection{Transfers Between Fluids to mimic convection parameterisation}

We seek to demonstrate that the numerical method is stable in the
presence of large transfers between fluids and we choose transfer
terms that lead to warm rising air in fluid one and the stable and
descending air in fluid zero. We examine stability, energy conservation
and comparison with the single fluid case. These simulations start
with no fluid one (all the mass initially in fluid zero). Whenever
mass is transferred into fluid one it takes its properties with it
so the solution should be identical to the single fluid case and energy
should be conserved. \textcolor{blue}{In order to maintain stability,
diffusion between fluids of $K_{\sigma}=100\ \text{m}^{2}\text{s}^{-1}$
and drag with coefficient of $C_{D}=1/2$, $r_{\min}=1\ \text{m}$
and $r_{\max}=2000\ \text{m}$ are used.}

\subsubsection{Transfers based on buoyancy perturbations}

We test the numerical solution using the mass transfer terms associated
with buoyancy perturbations from eqns (\ref{eq:thetaTransfer01},\ref{eq:thetaTransfer10}).
To test that warm air is transferred from fluid zero to fluid one,
we initialise the simulation with $\sigma_{0}=0$ everywhere and the
warm perturbation in fluid zero only. The solutions for $\sigma_{1}$,
$\theta_{0,1}$ and $\mathbf{u}_{0,1}$ after 1000\,s are shown in
figure \ref{fig:massTransfer}\textcolor{blue}{{} using diffusivities
$K_{\theta}=10^{6}\ \text{m}^{2}\text{s}^{-1}$ (top left) and $K_{\theta}=10^{5}\ \text{m}^{2}\text{s}^{-1}$
(bottom left). Figure \ref{fig:massTransfer} confirms that $\theta_{i}$
and $\mathbf{u}_{i}$ are similar in each fluid and also similar to
the solutions in figure \ref{fig:identicalParts}. Using $K_{\theta}=10^{6}\ \text{m}^{2}\text{s}^{-1}$,
warm fluid is completely transferred to fluid one however using $K_{\theta}=10^{5}\ \text{m}^{2}\text{s}^{-1}$,
the transfer is partial.}

The $\ell_{2}$ errors shown in figure \ref{fig:massTransfer} show
the root mean square difference between the single fluid $\theta$
and the mean $\theta$ across all partitions ($\sum_{i}\sigma_{i}\rho_{i}\theta_{i}/\sum_{i}\sigma_{i}\rho_{i}$)
normalised by the root mean square single fluid $\theta$. Fluid one
is initialised with no mass and without a warm bubble. The zero mass
in fluid one means that once mass is transferred to fluid one it should
have identical properties to fluid zero. This does not happen exactly;
the RMS errors are low \textcolor{blue}{but not zero}. These RMS errors
can be compared with the RMS differences between the simulations initialised
with identical properties in both fluids and the single fluid in figure
\ref{fig:identicalParts}. These have RMS differences of $7.7\times10^{-14}$.
Therefore the simulation of a zero mass fluid with different initial
conditions is introducing numerical error \textcolor{blue}{although
these errors are not obvious when comparing the results of figure
\ref{fig:massTransfer} with those in figure \ref{fig:identicalParts}
by eye}. 

\begin{figure*}
\includegraphics[width=1\linewidth]{figures/massTransfer}

\caption{Rising bubble solutions of $\sigma_{1}$, $\theta_{0,1}$ and $\mathbf{u}_{0,1}$
after 1000\,s with mass transfers based on $K_{\theta}\nabla^{2}\theta/\theta$
and based on $\nabla_{h}\cdot\mathbf{u}$ and $\mathbf{u}\cdot\mathbf{g}$.
\textcolor{blue}{All simulations use $K_{\sigma}=100\ \text{m}^{2}\text{s}^{-1}$
and drag with coefficient of $C_{D}=1/2$, $r_{\min}=1\ \text{m}$
and $r_{\max}=2000\ \text{m}$. }The $\ell_{2}$ error norms are the
normalised root mean square difference between the single fluid $\theta$
and the mean $\theta$ across all partitions ($\sum_{i}\sigma_{i}\rho_{i}\theta_{i}/\sum_{i}\sigma_{i}\rho_{i}$).
\label{fig:massTransfer}}
\end{figure*}

The changes in energy for both solutions with mass transfer based
on buoyancy perturbations are shown in figure \ref{fig:massTransferEnergy}.
The energy loss is very similar to the single fluid case (figure \ref{fig:energy1part}). 

\begin{figure}
\noindent \begin{centering}
\includegraphics[width=1\linewidth]{figures/massTransferEnergy}
\par\end{centering}
\caption{Normalised energy changes for the rising bubble solutions with mass
transfer based on $K_{\theta}\nabla^{2}\theta/\theta$ and based on
$\nabla_{h}\cdot\mathbf{u}$ and $\mathbf{u}\cdot\mathbf{g}$. Dashed
lines show negative changes (all changes are negative). \label{fig:massTransferEnergy}}
\end{figure}

\subsubsection{Transfers based on horizontal divergence and vertical velocity}

The results of the simulation using mass transfer based on horizontal
divergence and vertical velocity from eqns (\ref{eq:divTransfer01},\ref{eq:divTransfer10})
are shown on the top right of figure \ref{fig:massTransfer}. Again,
since fluid one is initially empty, its initial properties should
not influence the final solution. Instead, properties are transferred
from fluid zero and the two fluids evolve in the same way and the
solutions are very similar to the single fluid simulation, with normalised
RMS differences of $1.9\times10^{-7}$ \textcolor{blue}{which again
is not big enough to visually see differences between the results
in figures \ref{fig:massTransfer} and \ref{fig:identicalParts} }.
Using mass transfer based on horizontal divergence leads to larger
differences from the single fluid case but more mass is transferred
so this is not a disadvantage of mass transfer based on horizontal
divergence. However transfer based on horizontal divergence leads
to transfer of the fluid that is behind the warm air and not the warm
air itself. The warm air itself actually initially expands to satisfy
the perfect gas law \textcolor{blue}{and so is not transferred}. 

Using mass transfer based on horizontal divergence, the drop in energy
is again very similar to the single fluid case (comparing figures
\ref{fig:energy1part} and \ref{fig:massTransferEnergy}). 

\subsubsection{\textcolor{blue}{Transfers based on buoyancy perturbations, horizontal
divergence and vertical velocity}}

\textcolor{blue}{Finally we use both types of mass transfer in the
simulation shown in the bottom right of figure \ref{fig:massTransfer}.
One might expect the fluid transferred to be the sum of the fluids
transferred based on both processes seaparately but this is not the
case because of inconsistencies between the two transfer processes.
For example, $\nabla^{2}\theta$ is positive behind the warm anomaly
so fluid is transferred back to fluid 1, despite the horizontal convergence
and rising air. It does not appear useful to use both types of transfer. }

\section{Summary, Conclusions and Further Work}

A stable numerical method is presented for solving the conditionally
averaged equations of motion for representing atmospheric convection
with two fluids, one to represent stable, environmental air and the
other to represent buoyant plumes. This builds on traditional mass
flux schemes by solving equations for mass, temperature and momentum
both inside and outside the plumes and so net mass transfer by convection
can be represented. Our numerical method would also be suitable for
the similar extended EDMF scheme \citep{TKP+18} and would allow net
mass transport by convection. Transfers of mass between the fluids
are proposed for two purposes:
\begin{enumerate}
\item The conditionally averaged equations with a single pressure are ill-posed
so transfer terms are formulated to \textcolor{blue}{represent drag
and entrainment between fluids} to ensure that the different fluids
\textcolor{blue}{do not diverge}. 
\item Transfer terms are proposed so that well resolved convection is transferred
to the buoyant fluid and stable or sinking air is transferred to the
stable fluid. These are based on buoyancy anomalies (measured by the
Laplacian of $\theta$) and based on horizontal divergence.
\end{enumerate}
The transfer terms are applied explicitly to the individual partition
continuity equations (the transport equations for $\sigma_{i}\rho_{i}$)
and limited to avoid negative mass in each fluid. These mass transfer
terms also appear in the temperature and momentum equations since
the transferred mass takes its other properties with it. In these
equations the transfer terms are treated implicitly as they can be
very large. The numerical treatment of these transfer terms ensures
boundedness and mass, momentum and internal energy conservation on
transfer.

A semi-implicit finite volume method is used to solve the equations
of motion in advective form which ensures boundedness of the fluid
fraction, $\sigma_{i}$, and \textcolor{blue}{removes time-step restrictions
based on the speed of sound}. This aspect of the solution entails
substituting both (or all) momentum equations into the continuity
equation rather than just the mean flow momentum equation as is done
in semi-implicit weather forecast models. Without this numerical treatment,
net mass transport by convection would trigger acoustic waves that
are not handled by the implicit part of the model which would lead
to instability for moderate time-steps. 

\textcolor{blue}{The results presented in this paper use explicit
Eulerian advection schemes and have advective Courant number limits
of 0.5. This Courant number restriction could be raised with the use
of semi-Lagrangian or implicit advection. This could be important
for large updrafts in plumes. }

Results are presented of a well resolved rising warm bubble \textcolor{blue}{simulated
with the two fluid equations. Without transfer terms the model is
unstable since divergence in each fluid is not controlled. The model
is stabilised by adding either lateral entrainment as diffusion of
mass between the fluids or may adding drag between the fluids. Transfer
terms are also included to move the warm rising air from the stable
fluid to the buoyant fluid.}

\textcolor{blue}{This paper has not answered the question of how much
stabilisation of the conditionally averaged equations is necessary,
just that two different techniques can stabilise a model of the conditionally
averaged equations. Numerical analysis to find the minimum necessary
stabilisation will be the subject of a future publication. }

The formulation of transfer terms in order to represent sub-grid convection
is the subject of future work. These transfer terms should depend
on sub-grid variability of the primitive variables in each fluid.
\textcolor{blue}{For representing unresolved flow, parameterisations
for sub-filter-scale fluxes will also be essential.}

\section*{Acknowledgements}

Many thanks to Peter Clark, John Thuburn and Chris Holloway for valuable
discussions and proof reading. Thanks to the NERC/Met Office Paracon
project. We acknowledge funding from the RevCon Paracon project NE/N013743/1.

\bibliographystyle{abbrvnat}
\bibliography{numerics}

\appendix

\section{Spatial Discretisation \label{appx:modelSpace}}

\textcolor{blue}{This appendix defines the spatial discretisation
that is used. This is similar to \citet{WS14} which was used for
single fluid equations. None of the spatial discretisation described
is specific for conditionally averaged equations. The spatial discretisation
is general for arbitrarily structured meshes (for example including
hexagonal prisms, cut cells and refinement) but the meshes used in
this study or all fully structured. }

\subsection{Reconstruction of velocity fields at cell centres and faces from
face normals}

\textcolor{blue}{The model uses a C-grid so normal components of the
velocity (volume fluxes) are stored at cell faces. Full velocity fields
are needed in the non-linear advection term, $\mathbf{u}\cdot\nabla\mathbf{u}$,
of the momentum equation (\ref{eq:condMomAdv}). }The volume flux
across each face is:
\begin{equation}
U_{i}=\mathbf{u}_{fi}\cdot\mathbf{S}_{f}
\end{equation}
where $\mathbf{u}_{f}$ is the  velocity at the face and $\mathbf{S}_{f}$
is the face area vector. The face velocity is interpolated from the
cell centre velocity using linear interpolation:
\begin{equation}
\mathbf{u}_{fi}=\lambda\mathbf{u}_{ci}+(1-\lambda)\mathbf{u}_{Ni}\label{eq:linearInterpu}
\end{equation}
where $\mathbf{u}_{ci}$ is the cell centre velocity of the cell that
owns face $f$, $\mathbf{u}_{Ni}$ is the cell centre velocity of
the cell on the other side of face $f$ and $\lambda$ is the linear
interpolation weight. The face area vector, $\mathbf{S}_{f}$ is normal
to the face, has the area of the face and points from the owner cell
to the neighbour cell. The cell centre velocity is reconstructed from
surrounding values of $U_{i}$ using the standard OpenFOAM \texttt{fvc::reconstruct}:
\begin{equation}
\mathbf{u}_{ci}=\left(\sum_{f\in c}\mathbf{\hat{S}}_{f}\mathbf{S}_{f}^{T}\right)^{-1}\sum_{f\in c}U_{i}\mathbf{\hat{S}}_{f}\label{eq:recontructU}
\end{equation}
where the hat denotes the unit vector and the notation $f\in c$ means
all the faces, $f$ of cell $c$. Note that $\sum_{f\in c}\mathbf{\hat{S}}_{f}\mathbf{S}_{f}^{T}$
is a tensor defined on each cell that depends only on the mesh and
its inverse multiplies the vector $\sum_{f\in c}U_{i}\mathbf{\hat{S}}_{f}$
for each cell. 

\textcolor{blue}{This reconstruction is first-order accurate on arbitrary
meshes and second-order accurate on uniform structured meshes. It
simplifies to simple averaging of nearest neighbours on a uniform
structured mesh of hexahedra.}

\subsection{Non-Linear Advection}

\textcolor{blue}{This section is again concerned with the discretisation
of the non-linear advection term, $\mathbf{u}\cdot\nabla\mathbf{u}$,
of the momentum equation (\ref{eq:condMomAdv}). The non-linear advection
described is not optimal but yields satisfactory solutions using existing
OpenFOAM operators to implement a C-grid on an unstructured mesh. }

The conditionally averaged Euler equations are solved in advective
form so that they are defined where $\sigma_{i}=0$ and so that a
bounded advection scheme can be applied to $\sigma_{i}$. The finite
volume technique most naturally lends itself to solving equations
in flux form rather than advective form and so the non-linear advection
term of the momentum equation is calculated as:
\begin{equation}
\mathbf{u}_{i}\cdot\nabla\mathbf{u}_{i}=\nabla\cdot\left(\mathbf{u}_{i}^{T}\mathbf{u}_{i}\right)-\mathbf{u}_{i}\nabla\cdot\mathbf{u}_{i}.
\end{equation}
This quantity is calculated at cell centres and then linearly interpolated
onto faces. Both terms use Gauss's divergence theorem:
\begin{equation}
\mathbf{u}_{i}\cdot\nabla\mathbf{u}_{i}\approx\frac{1}{V}\left(\sum_{f\in c}\mathbf{u}_{ai}U_{i}-\mathbf{u}_{ci}\sum_{f\in c}U_{i}\right)\label{eq:nonLinearAdvect}
\end{equation}
where $V$ is the cell volume and $\mathbf{u}_{ai}$ is the velocity
interpolated from cell centres to faces using the OpenFOAM \textcolor{blue}{(second-order)}
linear upwind advection scheme. The prognostic variable is $U_{i}$
so the non-linear advection term from eqn (\ref{eq:nonLinearAdvect})
is linearly interpolated onto faces and then the dot product is taken
with $\mathbf{S}_{f}$. This advection is not bounded and requires
more interpolations and reconstructions than are usually used for
C-grid advection of velocity. It is used because it makes implementation
of C-grid advection straightforward in OpenFOAM. \textcolor{blue}{The
linear-upwind advection can be replaced by linear advection which
gives better energy conservation but the results are a bit noisy. }

\subsubsection{Pressure Gradient including the $\theta_{i}$ pre-factor \label{subsec:gradP}}

The pressure gradient term, $c_{p}\theta_{i}\nabla\pi$, \textcolor{blue}{of
the momentum equation (\ref{eq:condMomAdv})} needs to be calculated
on faces in the normal direction to the face. Since we assume an orthogonal
mesh this can simply be calculated as the difference in pressure between
the cells either side of the face:
\begin{equation}
c_{p}\theta_{i}\nabla\pi\cdot\mathbf{S}_{f}=c_{p}\theta_{fi}\nabla_{S}\pi\approx c_{p}\theta_{fi}\frac{\pi_{c}-\pi_{N}}{\delta}\ |\mathbf{S}_{f}|\label{eq:gradP}
\end{equation}
where $\theta_{fi}$ is $\theta_{i}$ linearly interpolated from cell
centres to faces, $\nabla_{S}\pi=\nabla\pi\cdot\mathbf{S}_{f}$, $\pi_{c}$
and $\pi_{N}$ are the values of Exner pressure in the cells either
side of face $f$ and $\delta$ is the distance between the cell centres.
The interpolation of $\theta$ in this term means that the discretisation
uses Lorenz staggering in the vertical.

\subsection{Transfer Terms in the Momentum Equation}

The transfer terms on the right hand side of the momentum equation
(\ref{eq:condMomAdv}) are calculated at cell centres and so need
to be linearly interpolated onto faces, and their dot products with
$\mathbf{S}_{f}$ are taken in order to calculate the rate of change
of $U_{i}$. This is not done using conservative interpolation so
momentum could be \textcolor{blue}{created or} destroyed when mass
is transferred. 

\subsection{Advection of $\sigma_{i}\rho_{i}$\label{subsec:vanLeerContinuity}}

\textcolor{blue}{Here we consider the discretisation of the advection
term of the continuity equation (\ref{eq:condCont}). }The mass transfer
terms that appear in the $\theta$ $\theta$ and momentum equations
(\ref{eq:condMomAdv},\ref{eq:condThetaAdv}) involve division by
$\sigma_{i}\rho_{i}$ so for stability, $\sigma_{i}\rho_{i}$ should
remain positive. Therefore $\sigma_{i}\rho_{i}$ should be advected
using a monotonic scheme. However for energy conservation and for
consistency between the continuity and pressure equation (which uses
the continuity equation, see section \ref{subsec:Helmholtz}), $\rho_{i}$
should be advected using centred differencing. The advection term
of eqn \ref{eq:condCont-1} is therefore discretised using Gauss's
divergence theorem as:
\begin{equation}
\nabla\cdot(\sigma_{i}\rho_{i}\mathbf{u}_{i})\approx\frac{1}{V}\sum_{f\in c}\sigma_{ia}\rho_{if}U_{f}
\end{equation}
where $\rho_{if}=\lambda\rho_{ic}+(1-\lambda)\rho_{iN}$ using the
same interpolation weights and notation as for eqn (\ref{eq:linearInterpu})
and where $\sigma_{ia}$ is interpolated from cell centre values of
$\sigma_{i}$ to faces using the monotonic OpenFOAM van-Leer advection
scheme:
\begin{equation}
\sigma_{ia}=\sigma_{iu}+\phi\left(\sigma_{if}-\sigma_{iu}\right)
\end{equation}
where $\sigma_{iu}$ is the value of $\sigma_{i}$ in the upwind cell,
$\sigma_{if}$ is the linearly interpolated value and $\phi$ is the
van-Leer limiter function:
\begin{equation}
\phi=\frac{r+|r|}{1+|r|}\ ,\ \ r=2\frac{\left(\mathbf{x}_{d}-\mathbf{x}_{u}\right)\cdot\nabla_{u}\sigma_{i}}{\sigma_{id}-\sigma_{iu}}-1
\end{equation}
where $\mathbf{x}_{u}$ and $\mathbf{x}_{d}$ are the locations of
the upwind and downwind cell centres, $\sigma_{id}$ is the value
of $\sigma_{i}$ in the downwind cell and $\nabla_{u}\sigma_{i}$
is the gradient of $\sigma_{i}$ calculated in the upwind cell using
Gauss's divergence theorem. This discretisation is only monotonic
when $\rho_{i}$ is sufficiently smooth in the direction of flow which
is usually achieved in low Mach number flows. 

\subsection{Advection of $\theta_{i}$\label{subsec:thetaAdvect}}

\textcolor{blue}{Here we consider the advection term in the temperature
equation (\ref{eq:condThetaAdv}).} For consistency with advection
in the momentum equation (eqn \ref{eq:nonLinearAdvect}), $\theta$
advection is calculated in advective form using finite volume operators
(using Gauss's theorem):
\begin{equation}
\mathbf{u}_{i}\cdot\nabla\theta_{i}\approx\frac{1}{V}\left(\sum_{f\in c}\theta_{ai}U_{i}-\theta_{i}\sum_{f\in c}U_{i}\right)\label{eq:thetaAdvect}
\end{equation}
where $\theta_{a}$ is interpolated from cell centres to faces using
the OpenFOAM van-Leer advection scheme as for $\sigma_{i}$.
\end{document}
